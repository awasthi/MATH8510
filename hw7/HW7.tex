\documentclass{amsart}

% PACKAGES

\usepackage{amsmath}
\usepackage{amsfonts}
\usepackage{amssymb,enumerate}
\usepackage{amsthm,stmaryrd}
\usepackage[all]{xy}
\usepackage{hyperref}

%\theoremstyle{definition}
%\newtheorem{exer}{Exercise}

\newcommand{\bbr}{\mathbb{R}}
\newcommand{\bbc}{\mathbb{C}}
\newcommand{\bbz}{\mathbb{Z}}
\newcommand{\bbq}{\mathbb{Q}}
\newcommand{\bbn}{\mathbb{N}}
\newcommand{\be}{\mathbf{e}}
\newcommand{\ba}{\mathbf{a}}
\newcommand{\fm}{\mathfrak{m}}
\newcommand{\Hom}{\operatorname{Hom}}
\renewcommand{\ker}{\operatorname{Ker}}
\newcommand{\im}{\operatorname{Im}}
\newcommand{\xra}{\xrightarrow}
\newcommand{\wti}{\widetilde}

\theoremstyle{plain}
\newtheorem{lem}{Lemma}
\newtheorem{cor}[lem]{Corollary}
\newtheorem{prop}[lem]{Proposition}
\newtheorem{thm}[lem]{Theorem}
\newtheorem{conj}[lem]{Conjecture}
\newtheorem{intthm}{Theorem}
\renewcommand{\theintthm}{\Alph{intthm}}

\theoremstyle{definition}
\newtheorem{defn}[lem]{Definition}
\newtheorem{ex}[lem]{Example}
\newtheorem{question}[lem]{Question}
\newtheorem{questions}[lem]{Questions}
\newtheorem{problem}[lem]{Problem}
\newtheorem{disc}[lem]{Remark}
\newtheorem{rmk}[lem]{Remark}
\newtheorem{construction}[lem]{Construction}
\newtheorem{notn}[lem]{Notation}
\newtheorem{fact}[lem]{Fact}
\newtheorem{para}[lem]{}
\newtheorem{exer}[lem]{Exercise}
\newtheorem{remarkdefinition}[lem]{Remark/Definition}
\newtheorem{notation}[lem]{Notation}
\newtheorem{step}{Step}
\newtheorem{convention}[lem]{Convention}
\newtheorem*{Convention}{Convention}
\newtheorem{assumption}[lem]{Assumption}

\newcommand{\fmn}{F^{m\times n}}
\newcommand{\fnn}{F^{n\times n}}
\newcommand{\col}{\operatorname{Col}}
\newcommand{\row}{\operatorname{Row}}
\newcommand{\Span}{\operatorname{Span}}	
\newcommand{\rank}{\operatorname{rank}}	
\newcommand{\OO}[1]{\mathcal{O}_{#1}}
\begin{document}

\noindent MATH 8510, Abstract Algebra I \\
Fall 2016\\
Exercises 7\\
Name: Shuai Wei\\
Collaborator: Dao Zhou Zhu
\

%\noindent
%Throughout this homework set, let $F$ be a field
%
%
%\

\begin{exer}[4.4.1]
Let $G$ be a group.
\begin{enumerate}[(a)]
\item Let $\tau\in\operatorname{Aut}(G)$, and let $\sigma_g\in\operatorname{Inn}(G)$ be conjugation by $g\in G$.
Prove that $\tau\sigma_g\tau^{-1}=\sigma_{\tau(g)}$.
\begin{proof}
	Since $\tau \in \operatorname{Aut}(G)$, it is a homomorphism. \\
	Then $\forall \ x,y \in G$, $\tau(xy) = \tau(x)\tau(y)$ and $\tau(x^{-1}) = \tau(x)^{-1}$.\\
	Then $\forall x \in G$,
	\begin{align*}
	  \tau\sigma_g\tau^{-1}(x)&= \tau g\tau^{-1}(x)g^{-1}\\
	  						  &=\tau(g\tau^{-1}(x)g^{-1})\\
	  						  &=\tau(g)\tau\left(\tau^{-1}(x)\right)\tau\left(g^{-1}\right)\\
	  						  &=\tau(g)x\tau(g)^{-1}\\
	  						  &=\sigma_{\tau(g)}(x)
  \end{align*}
  Thus, 
  \[\tau\sigma_g\tau^{-1} = \sigma_{\tau(g)}.\]
\end{proof}

\item Prove that $\operatorname{Inn}(G)\unlhd\operatorname{Aut}(G)$.
  \begin{proof}
	First we show $\operatorname{Inn}(G) \leq \operatorname{Aut}(G)$.\\
	\begin{enumerate}
		\item
			By the definition of $\operatorname{Inn}(G)$ and $\operatorname{Aut}(G)$, we have
			\[\operatorname{Inn}(G) \subset \operatorname{Aut}(G)\]
		\item
		  $e_G \in G$ is the identity, $\sigma_{e_G} \in \operatorname{Inn}(G)$.\\
		 So 
		 \[\operatorname{Inn}(G) \neq \emptyset.\]
		\item
		  Let $\sigma_f,\sigma_h \in \operatorname{Inn}(G) \subset \operatorname{Aut}(G)$.\\
		We first compute $\sigma_k^{-1}(x)$ for $k,x\in G$.\\
		Let $k, x \in G$, then 
		\[\sigma_k(x) = kxk^{-1}.\]
		Then
		\[x = k^{-1}\sigma_k(x)k.\]
		So 
		\[\sigma_k^{-1}(x) = k^{-1}xk.\]
		Thus, 
		\[\sigma_k^{-1} = \sigma_{k^{-1}}.\]
		Then $\forall\ f,h,x \in G$, \\
		we have $fh^{-1} \in G$ and 
		\begin{align*}
		  \sigma_f\sigma_h^{-1}(x) &= \sigma_f(h^{-1}xh) \\
		  						   &= f(h^{-1}xh)f^{-1} \\
		  						   &= (fh^{-1})x(fh^{-1})^{-1} \\
		  						   &=\sigma_{fh^{-1}}(x).	
		\end{align*}
		$x\in G$ is arbitrary, so
		\[\sigma_f\sigma_h^{-1} = \sigma_{fh^{-1}}.\]
		Since $fh^{-1} \in G$,
		\[\sigma_f\sigma_h^{-1}  = \sigma_{fh^{-1}} \in \operatorname{Inn}(G).\]
	\end{enumerate}
	Thus, 
	\[\operatorname{Inn}(G) \leq \operatorname{Aut}(G). \]
  	Moreover, in (a), we have shown that $\forall\ \tau \in \operatorname{Aut}(G)$ and $\sigma_g \in \operatorname{Inn}(G)$,\\ 
  	we have $\tau\sigma_g\tau^{-1} = \sigma_{\tau(g)}$.\\
  	Besides, since $\tau \in \operatorname{Aut}(G)$ and $g\in G$, 
  	\[\tau(g) \in G.\]
  	So
  	\[\tau\sigma_g\tau^{-1} = \sigma_{\tau(g)} \in \operatorname{Inn}(G).\]
  	Therefore, 
  	\[\operatorname{Inn}(G) \unlhd \operatorname{Aut}(G). \]
  \end{proof}
\end{enumerate}
\end{exer}

\begin{exer}[4.4.2]
Let $G$ be a group.
\begin{enumerate}[(a)]
\item Prove that if $G$ is  abelian  of order $pq$ where $p$ and $q$ are distinct primes, then $G$ is cyclic.
	\begin{proof}	
	$|G| = pq$ and $p,q$ are primes, then by Cauchy's theorem,\\
	there exists $x,y \in G$ such that $|x|=p$ and $|y|=q$.\\
	Then $x^p = e_G$ and $y^q = e_G$.\\
	Since $G$ is abelian,\\
	\[(xy)^{pq} = (x^p)^q (y^q)^p = (e_G)^p(e_G)^q = e_G.\]
	So 
	\[|xy| \mid  {pq}.\]
	Then 
	\[|xy| = 1 \text{ or } p \text{ or }  q \text{ or } pq.\]
	\begin{enumerate}
	  \item If $|xy| = 1$, then $xy = e_G$, so $x^{-1} = y$.\\
	  	Let $g \in \langle y\rangle$, then there exists $n \in \bbn$ such that $g = y^n = x^{-n}$.\\
	  	So 
	  	\[\langle y \rangle \subset \langle x \rangle.\]
		Similarly, we have 
		\[ \langle x \rangle \subset \langle y \rangle.\]
		Thus,
	  	\[\langle x \rangle = \langle y \rangle.\]
		Then
		\[p = q,\]
		which is a contradiction since $p,\ q$ are distinct primes.\\
		Therefore, $|xy| \neq 1$.
	  \item
	  	If $|xy| = p$, then given $G$ is abelian,
	  	\[(xy)^p = x^py^p = y^p = e_G.\]
	  	Since $|y| = q$,
	  	\[q\mid p.\]
	  	which is a contradiction since $p, q$ are arbitrary primes.\\
	  	Therefore, $|xy| \neq p$.
	  	\item
	  	 Similarly, we have $|xy| \neq q$.
	  \end{enumerate}
	  Hence, \[|xy| = pq.\]
	As a result, $G$ is cyclic.
	\end{proof}

\item Prove that if $|G|=15$, then $G\cong\mathbb Z/15\mathbb Z$.
  	\begin{proof}
  	  Since $|G| = 3\times 5$ and $3 \nmid (5-1)$, by what we have shown in class, we have $G$ is abelian.\\
  	  Then by part (a), 3 and 5 are primes, so we also have $G$ is cyclic.\\
  	  Since $|G| = 15 < \infty$, 
  	  \[G \cong\mathbb Z/15\mathbb Z. \]
  	\end{proof}
\end{enumerate}
\end{exer}

\begin{exer}[4.5.22]
Prove that if $G$ is a group with $|G|=132$, then $G$ is not simple.
\begin{proof}
 Let $P \in Syl_{11}(G)$, $Q \in Syl_{3}(G)$ and $R \in Syl_{2}(G)$.\\
 Since $132 = 2^2\times 3 \times 5$, by Sylow Theorem 4.5.1, \\
 we have$|P| = 11, |Q| = 3$ and $|R| = 4$.\\
Suppose $P \ntrianglelefteq G$ and $Q \ntrianglelefteq G$ and $R \ntrianglelefteq G$.\\
Then $n_{11} \neq 1$ and $n_{3} \neq 1$ and $n_{2} \neq 1$.
\begin{enumerate}
	\item
	  By Sylow Theorem 4.5.4, we have $n_{11} \equiv (1 \mod{11})$ and $n_{11}\mid 12$.\\
	    $n_{11}\mid 12$, so $n_{11} = 1$ or $2$ or $3$ or $4$ or $6$ or $12$.\\
		Besides, $n_{11} \equiv (1 \mod{11})$, so $n_{11} = 1$ or $12$.\\	
		We already know $n_{11} \neq 1$.\\
		So $n_{11} = 12$.\\
		Then by Lemma 4.5.7, we have
		\[\left|\{x\in G: |x| = 11\}\right| = 12\times (11-1) = 120. \]
	\item
	  By Sylow Theorem 4.5.4, we have $n_{3} \equiv (1 \mod{3})$ and $n_{3}\mid 44$.\\
		$n_{11}\mid 44$, so $n_{3} = 1$ or $2$ or $22$ or $4$ or $11$ or $44$.\\
		Besides, $n_{3} \equiv (1 \mod{3})$, so $n_{3} = 1$ or $4$.\\	
		We already know $n_{3} \neq 1$.\\
		So $n_{3} = 4$.\\
		Then by Lemma 4.5.7, we have
		\[\left|\{x\in G: |x| = 3\}\right| = 4\times (3-1) = 8.\]
	\item
	  	Since $n_2 \neq 1$ and $n_2 \in \bbn$, we have $n_2 \geq 2$.\\
		Then, for distinct $P,P^{\textprime} \in Syl_2(G)$, $P \cap P^{\textprime}$ might be non-trivial.\\
		\[\left|\{x\in G: |x| = 2\}\right| \geq |P\setminus\{e\}| + 1= 3 +1 =4.\]
\end{enumerate}
		Since $|e| = 1$,
		\[|G| \geq 120+8+4+1 = 133,\]
		which is a contradiction since $|G| = 132$. \\
		Thus, either $P$ or $Q$ or $R$ is a normal subgroup of $G$.\\
      	We already show 
      	\[1<|P|,|Q|,|R|<|G|.\]
        So either $P$ or $Q$ or $R$ is a non-trivial normal subgroup of $G$.\\
		So $G$ is not simple.
 
\end{proof}
\end{exer}

\begin{exer}[4.5.24]
Prove that if $G$ is a group with $|G|=231$, then $Z(G)$ contains a Sylow 11-subgroup of $G$.
\end{exer}
\begin{proof}
	$|G| = 3 \times 7 \times 11$, and $n_{11} = |Syl_{11}(G)|$.\\
	Then by Sylow Theorem 4.5.4, we have $n_{11} \equiv (1 \mod{11})$ and $n_{11} | 21$.\\
	$n_{11} | 21$, so $n_{11} = 1$ or $3$ or $7$ or $21$.\\
	Besides, $n_{11} \equiv (1 \mod{11})$, so $n_{11} = 1$.\\
	Let $H \in Syl_{11}(G)$, then by Sylow Theorem 4.5.4, we have $|H|=11$ and by Corollary 4.5.5, we have $H \unlhd G$. \\
	11 is a prime, so $H$ is cyclic according to what we have shown in class.\\
	Since $C_G(H) \leq G$ and $H \leq G$, and $H \subset C_G(H)$, 
	\[H \leq C_G(H).\]
	So
	\[|H| \mid |C_G(H)|. \]
	Since $|H| = 11 = |G|/21$,
	\[(|G|/21) \mid |C_G(H)|.\]	
	Then 
	\[|G| \mid (21C_G(H)|) .\]
	Then 
	\[\left(|G|/|C_G(H)|\right) \mid 21.\]
	Moreover, since $H\unlhd G$, by Proposition 4.4.2(d), we have $G/C_G(H)$ is isomorphic to a subgroup of $\operatorname{Aut}(H)$.\\
	So 
	\[\left(|G|/|C_G(H)|\right) \mid |\operatorname{Aut}(H)|.\]
	Since $H$ is a cyclic group of order 11, $H\cong \bbz/11\bbz$.\\
	Then 
	\[\operatorname{Aut}(H) \cong \operatorname{Aut}(\bbz/11\bbz).\]
	Then by Proposition 4.4.5, 
	\[|\operatorname{Aut}(H)| = \varphi(11) = 10,\]
	where $\varphi$ is $Euler \varphi$-$function$.\\
	So 
	\[\left(|G|/|C_G(H)|\right) \mid 10.\]\\
	The common factor(s) of $10$ and $21$ is just 1.\\
	So 
	\[|G|/|C_G(H)| = 1.\]
	Therefore,
	\[C_G(H) = G.\]
	Hence,
	\[H \subset Z(G).\]
	As a result, we have $Z(G)$ contains a Sylow 11-subgroup of $G$.
\end{proof}

\end{document}
