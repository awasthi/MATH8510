\documentclass{amsart}

% PACKAGES

\usepackage{amsmath}
\usepackage{amsfonts}
\usepackage{amssymb,enumerate}
\usepackage{amsthm,stmaryrd}
\usepackage[all]{xy}
\usepackage{hyperref}

%\theoremstyle{definition}
%\newtheorem{exer}{Exercise}

\newcommand{\bbr}{\mathbb{R}}
\newcommand{\bbc}{\mathbb{C}}
\newcommand{\bbz}{\mathbb{Z}}
\newcommand{\bbq}{\mathbb{Q}}
\newcommand{\bbn}{\mathbb{N}}
\newcommand{\be}{\mathbf{e}}
\newcommand{\ba}{\mathbf{a}}
\newcommand{\fm}{\mathfrak{m}}
\newcommand{\Hom}{\operatorname{Hom}}
\renewcommand{\ker}{\operatorname{Ker}}
\newcommand{\im}{\operatorname{Im}}
\newcommand{\xra}{\xrightarrow}
\newcommand{\wti}{\widetilde}

\theoremstyle{plain}
\newtheorem{lem}{Lemma}
\newtheorem{cor}[lem]{Corollary}
\newtheorem{prop}[lem]{Proposition}
\newtheorem{thm}[lem]{Theorem}
\newtheorem{conj}[lem]{Conjecture}
\newtheorem{intthm}{Theorem}
\renewcommand{\theintthm}{\Alph{intthm}}

\theoremstyle{definition}
\newtheorem{defn}[lem]{Definition}
\newtheorem{ex}[lem]{Example}
\newtheorem{question}[lem]{Question}
\newtheorem{questions}[lem]{Questions}
\newtheorem{problem}[lem]{Problem}
\newtheorem{disc}[lem]{Remark}
\newtheorem{rmk}[lem]{Remark}
\newtheorem{construction}[lem]{Construction}
\newtheorem{notn}[lem]{Notation}
\newtheorem{fact}[lem]{Fact}
\newtheorem{para}[lem]{}
\newtheorem{exer}[lem]{Exercise}
\newtheorem{remarkdefinition}[lem]{Remark/Definition}
\newtheorem{notation}[lem]{Notation}
\newtheorem{step}{Step}
\newtheorem{convention}[lem]{Convention}
\newtheorem*{Convention}{Convention}
\newtheorem{assumption}[lem]{Assumption}

\newcommand{\fmn}{F^{m\times n}}
\newcommand{\fnn}{F^{n\times n}}
\newcommand{\col}{\operatorname{Col}}
\newcommand{\row}{\operatorname{Row}}
\newcommand{\Span}{\operatorname{Span}}	
\newcommand{\rank}{\operatorname{rank}}	
\newcommand{\OO}[1]{\mathcal{O}_{#1}}
\begin{document}

\noindent MATH 8510, Abstract Algebra I \\
Fall 2016\\
Exercises 6-1\\
Name: Shuai Wei\\
Collaborator: DaoZhouZhu, XiaoyuanLiu, Ben\\
\

%\noindent
%Throughout this homework set, let $F$ be a field
%
%
%\

\begin{exer}[3.5.10]
Consider the alternating group $A_4$.
Prove that there is a chain of normal subgroups $\{(1)\}N_0\unlhd N_1\unlhd\cdots\unlhd N_k=A_4$ such that each quotient $N_i/N_{i-1}$ is abelian.
(This says that $A_4$ is \emph{solvable}.)

\

\noindent
Hint: 
Set $N=\{(1),(1\ 2)(3\ 4),(1\ 3)(2\ 4),(1\ 4)(2\ 3)\}\subseteq A_4$, and prove the following:
\begin{enumerate}[(a)]
\item 
Prove that $N\leq A_4$. 

\begin{proof}
  \begin{enumerate}[(i)]
	  \item It is obvious $N \subseteq A_4$
	  \item $N$ is not empty since $e_{A_4} = (1) \in N$.
	  \item Since $N=\{(1), (1~ 2)(3~ 4),(1~ 3)(2~ 4),(1~ 4)(2~ 3)\}$, \\
	  	$\left((1~ 2)(3~ 4)\right)\left((1~ 3)(2~ 4)\right) = (1~ 4)(2~ 3) \in N$, and \\
	  	$\left((1~ 2)(3~ 4)\right)\left((1~ 4)(2~ 3)\right) = (1~ 3)(2~ 4)\in N$, and\\
	  	$\left((1~ 3)(2~ 4)\right)\left((1~ 4)(2~ 3)\right) = (1~ 2)(3~ 4)\in N$.\\
		Similarly, \\
	  	$\left((1~ 3)(2~ 4)\right)\left((1~ 2)(3~ 4)\right) = (1~ 4)(2~ 3) \in N$, and \\
	  	$\left((1~ 4)(2~ 3)\right)\left((1~ 2)(3~ 4)\right) = (1~ 3)(2~ 4)\in N$, and\\
	  	$\left((1~ 4)(2~ 3)\right)\left((1~ 3)(2~ 4)\right) = (1~ 2)(3~ 4)\in N$.\\
	  	So $N$ is abelian.
	  \item $\left((1~ 2)(3~ 4)\right)\left((1~ 2)(3~ 4)\right) = (1) \in N$.\\
			$\left((1~ 3)(2~ 4)\right)\left((1~ 3)(2~ 4)\right) = (1) \in N$.\\
				$\left((1~ 4)(2~ 3)\right)\left((1~ 4)(2~ 3)\right) = (1) \in N$.\\
		So the inverses of $(1~ 2)(3~ 4),(1~ 3)(2~ 4)$ and $(1~ 4)(2~ 3)$ are themselves, respectively, which are obviously in $N$.
	\end{enumerate}
	Thus, $N$ is a subgroup of $A_4$
\end{proof}

\item 
Prove that $N\smallsetminus\{(1)\}=\{\tau\in A_4\mid|\tau|=2\}$.
\begin{proof}
  Let $D = \{(a~b~c),a,b,c \in \{1,2,3,4\}, a\neq b\neq c\}$. \\
  It is obvious that $(1) \not\in D$.\\
  Then $D \subsetneq A_4$ since $(a~b~c) = (a~c)(a~b)$.\\
  We have $(a~b~c)(a~b~c) = (a~c~b) \neq (1) $ and $(a~b~c)(a~b~c)(a~b~c) = (1)$, \\
  so $|(a~b~c)| = 3$. \\
  Beside, we have $|D| = \frac{4!}{3} = 8$. \\
  Since $|N| + |D| = 12 = |A_4|$ and $N \cap D = \emptyset $,\\
  $A_4 = N \cup D$.\\
  So we know all the elements with order 2 are in $A_4$. \\
  Moreover, all elements in $N$ has order 2 except element $(1)$.
  Thus,
  \[ N \smallsetminus \{(1)\} = \{\tau \in A_4~\mid~ |\tau| = 2\} .  \] 

\end{proof}

\item 
Prove that for all $\sigma\in A_4$, for all $\tau\in N\smallsetminus\{(1)\}$, the element $\sigma\tau\sigma^{-1}$ has order 2, so it is in $N$. 

\begin{proof}
  Suppose $|\sigma\tau\sigma^{-1}| = 1$, then $\sigma\tau\sigma^{-1} = (1)$. \\
  So $\sigma\tau = \sigma$.\\
  Then $\tau = (1)$, which is a contradiction since $\tau \in N\smallsetminus\{(1)\}$.\\
  So $|\sigma\tau\sigma^{-1}| > 1$.\\
  For all $\tau\in N\smallsetminus\{(1)\}$, we have $\tau \tau = (1)$ since the order of any element of $N\smallsetminus\{(1)\}$ is 2.\\
  For all $\sigma\in A_4$ and all $\tau\in N\smallsetminus\{(1)\}$,
  \begin{align*}
  	(\sigma\tau\sigma^{-1})(\sigma\tau\sigma^{-1}) &= \sigma\tau\tau\sigma^{-1} \\
  												   &= \sigma(1)\sigma^{-1} \\
  												   &= (1).
  \end{align*}
  	So the element $\sigma\tau\sigma^{-1}$ has order 2, and then $\sigma\tau\sigma^{-1} \in N$.\\
  	Thus, $N\unlhd A_4$.\\
	Since $|A_4/N| = \frac{|A_4|}{|N|} = 3$ by Lagrange Theorem, $A_4/N$ is simple and cyclic.\\
	Then $A_4/N$ is abelian.\\
	As a result, we have a trivial chain $N \unlhd A_4$.\\
  	Since $|N| = 4$, by Jordan-H\"older theorem, there is a chain of subgroups 
  	\[ \{(1)\} = N_0 \unlhd N_1...\unlhd N_k = N.\] 
  	$N$ is abelian by part (i), so $N_0,N_1, ..., N_{k-1}$ are also abelian since they are subgroups of $N$.\\ 
  	Thus, $N_1/N_0,N_2/N_1..,N/N_{n-1}$ are abelian.\\
	Combine the chain $\{(1)\} = N_0 \unlhd N_1...\unlhd N_k = N$ with the chain $N \unlhd A_4$,\\
	we get a new chain
	\[  N_0 \unlhd N_1...\unlhd  N \unlhd A_4,\]
	where $N_1/N_0, N_2/N_1..,N/N_{k-1}, A_4/N$ are abelian.\\
	Alternative soln: $\{(1)\} \unlhd N \unlhd A_4$.
 \end{proof}
  	
\end{enumerate}
\end{exer}

\begin{exer}[4.1.9]
Assume that $G$ acts transitively on a finite set $A$, and let $H\unlhd G$.
Note that $H$ also acts on $A$. 
Let $\OO{1}, \OO 2,\ldots,\OO r$ be the distinct orbits of $H$ on $A$
\begin{enumerate}[(a)]
\item Prove that $G$ permutes the sets $\OO{1}, \OO 2,\ldots,\OO r$ in the sense that for each $g\in G$
and each $i\in[r]=\{1,\ldots,r\}$, there is a $j$ such that $g\OO i=\OO j$ where $g\OO{}=\{ga\in A\mid a\in\OO{}\}$.
Prove that $G$ acts transitively on $\{\OO{1}, \OO 2,\ldots,\OO r\}$. Deduce that all orbits of $H$ on $A$ have the same cardinality.

\begin{proof}
  For each $g \in G$, and each $i\in [r] = \{1,\ldots,r\}$, there exists $a_i \in A$ such that $\OO{i} = H\cdot a_i = \{h\cdot a_i\in A \mid h\in H\}$.
  \begin{align*}
  	g\OO i &= g\{h\cdot a_i \in A \mid h\in H\} \\ 
  		  &= \{gh\cdot a_i \in A \mid h \in H\} \\
  		  & = \{h(g \cdot a_i) \in A \mid h \in H\} \\
  \end{align*}
  since $H \unlhd G$.\\ 
  Besides, $g \cdot a_i \in A$,so there exists $a \in A$ such that $g\cdot a_i = a$.\\
  There exists some $j \in [r]$ such that $a \in \OO j$ since $\OO 1,\OO 2, \ldots, \OO r$ are disjoint orbits of $H$ on $A$.\\
  Set $a_j = a$ for such $j \in [r]$ such that $a \in \OO j$ .\\
  Let $H\cdot a_j = \OO j = \{h\cdot a_j\in A \mid h\in H\}$ and then 
  \begin{align*}
  	 g\OO i  & = \{h(g \cdot a_i) \in A \mid h \in H\} \\
  	 		 & = \{ha_j \in A \mid h \in H\} \\
  			 & = \OO j 
  \end{align*}
  Next we show $G$ acts transitively on $\{\OO 1,\OO 2,\ldots,\OO r\}$.\\
  Since $G$ acts transitively on $A$, for any $j \in [r]$, there exists $g_j \in G$ such that $a_j = g_ja_1$, where $a_j, j\in [r]$ and $a_1$ is already defined by us. \\
  So for any $\OO j \in \{\OO 1,\OO 2,\ldots,\OO r\}$, 
  \begin{align*}
  		\OO j &= \{h\cdot a_j \in A \mid h\in H\} \\
			  &= \{h \cdot g_ja_1 \in A \mid h \in H \}\\
	    	  &= \{g_j h \cdot a_1 \in A\ \mid h \in H \}\\
			  &=g_j \{h\cdot a_1\in A \mid h\in H\} \\
	          &=g_j \OO 1
  \end{align*}
 Therefore, $G$ acts transitively on $\{\OO 1,\OO 2,\ldots,\OO r\}$.\\
 Since $\OO j = g_j \OO 1$ for any $j \in [r]$,\\
 we have for any $j \in [r]$,
 \[|\OO j| = |g_j \OO 1| = |\OO1|.\]
 Thus, we conclude that all orbits of H on A have the same cardinality.
\end{proof}


\item
Prove that if $a\in\OO 1$, then $|\OO 1|=[H\colon H\cap G_a]$, and prove that $r=[G\colon HG_a]$. 
\begin{proof}
We claim $H\cap G_a$ is the stablier of $a$ in H, namely, $H_a = H \cap G_a$.\\
\begin{enumerate}
	\item
Let $s \in H \cap G_a$, then $s \in H$ and $sa = a$ since $s \in G_a$.\\
So $s \in H_a$. \\
As are result, $H \cap G_a \subset H_a$.
\item
Let $h \in H_a$, then $h \in H$ and $ha = a$.\\
So $h \in G_a$ since $h \in H_a \subset H \unlhd G$.\\
Thus, $h \in H \cap G_a$.\\
Then $H_a \subset H \cap G_a$.
\end{enumerate}
Therefore, $H_a =H \cap G_a$.\\
Since $\OO{1}$ is one of the orbits of $H$ on $A$,\\
given $a \in \OO 1$, we have 
\begin{align*}
  |\OO 1| &= |\OO a|\\ 
  		 	  	&= [H:H_a] \\
  			 	&=[H:H\cap G_a]
\end{align*}
Then we will show $r=[G:HG_a]$. \\
Since $H\leq G$ and $G_a\leq G$, we have 
\begin{align*}
 	|HG_a| &= \frac{|H||G_a|}{|H\cap G_a|} \\
  		   	 &= \frac{|H||G_a|}{|H_a|} 
\end{align*}
Then (after proving $HG_a \leq G$!!!),
\begin{align*}
  [G:HG_a] &= \frac{|G|}{|HG_a|} \\
  		   &= \frac{|G||H_a|}{|H||G_a|} \\
\end{align*}
We have shown $|\OO 1|  = [H:H_a]$, so $|H|=|H_a||\OO 1|$.\\
Then
\begin{align*}
  [G:HG_a] &= \frac{|G|}{|G_a||\OO 1|} \\
\end{align*}
We know all orbits of $H$ on $A$ have the same cardinality. \\
So $r|\OO 1| = |A|$.\\
Then 
\begin{align*}
  [G:HG_a] &= \frac{|G|}{|G_a||A|}r 
\end{align*}
Since G acts transitively on A, for $a \in A$.
\[A = G\cdot a.\]
Then 
\begin{align*}
 	|A| &= |G\cdot a|\\
  		  &= [G:G_a] \\
  		&= \frac{|G|}{|G_a|}
\end{align*}
according to what we have shown in class.\\
Then 
\[ [G:HG_a] = \frac{|G|||G_a|}{|G_a||G|}r = r.\]
Namely, 
\[ r = [G:HG_a]\]
\end{proof}
\end{enumerate}
\end{exer}

\end{document}

















