\documentclass{amsart}

% PACKAGES

\usepackage{amsmath}
\usepackage{amsfonts}
\usepackage{amssymb,enumerate}
\usepackage{amsthm,stmaryrd}
\usepackage[all]{xy}
\usepackage{hyperref}

%\theoremstyle{definition}
%\newtheorem{exer}{Exercise}

\newcommand{\bbr}{\mathbb{R}}
\newcommand{\bbc}{\mathbb{C}}
\newcommand{\bbz}{\mathbb{Z}}
\newcommand{\bbq}{\mathbb{Q}}
\newcommand{\bbn}{\mathbb{N}}
\newcommand{\be}{\mathbf{e}}
\newcommand{\ba}{\mathbf{a}}
\newcommand{\fm}{\mathfrak{m}}
\newcommand{\Hom}{\operatorname{Hom}}
\renewcommand{\ker}{\operatorname{Ker}}
\newcommand{\im}{\operatorname{Im}}
\newcommand{\xra}{\xrightarrow}
\newcommand{\wti}{\widetilde}

\theoremstyle{plain}
\newtheorem{lem}{Lemma}
\newtheorem{cor}[lem]{Corollary}
\newtheorem{prop}[lem]{Proposition}
\newtheorem{thm}[lem]{Theorem}
\newtheorem{conj}[lem]{Conjecture}
\newtheorem{intthm}{Theorem}
\renewcommand{\theintthm}{\Alph{intthm}}

\theoremstyle{definition}
\newtheorem{defn}[lem]{Definition}
\newtheorem{ex}[lem]{Example}
\newtheorem{question}[lem]{Question}
\newtheorem{questions}[lem]{Questions}
\newtheorem{problem}[lem]{Problem}
\newtheorem{disc}[lem]{Remark}
\newtheorem{rmk}[lem]{Remark}
\newtheorem{construction}[lem]{Construction}
\newtheorem{notn}[lem]{Notation}
\newtheorem{fact}[lem]{Fact}
\newtheorem{para}[lem]{}
\newtheorem{exer}[lem]{Exercise}
\newtheorem{remarkdefinition}[lem]{Remark/Definition}
\newtheorem{notation}[lem]{Notation}
\newtheorem{step}{Step}
\newtheorem{convention}[lem]{Convention}
\newtheorem*{Convention}{Convention}
\newtheorem{assumption}[lem]{Assumption}

\newcommand{\fmn}{F^{m\times n}}
\newcommand{\fnn}{F^{n\times n}}
\newcommand{\col}{\operatorname{Col}}
\newcommand{\row}{\operatorname{Row}}
\newcommand{\Span}{\operatorname{Span}}	
\newcommand{\rank}{\operatorname{rank}}	
\newcommand{\OO}[1]{\mathcal{O}_{#1}}
\begin{document}

\noindent MATH 8510, Abstract Algebra I \\
Fall 2016\\
Exercises 11-2\\
Due date Thu 10 Nov 4:00PM

\

%\noindent
%Throughout this homework set, let $F$ be a field
%
%
%\

\begin{exer}
Let $R$ be a commutative ring with identity.
Prove that $R/N(R)$ has no non-zero nilpotent elements, that is, that $N(R/N(R))=0$.
\begin{proof}
  We have shown in Exercise 10-1$\#$2(a) that $N(R) \leq R$.\\
  By the Theorem 7.3.6., we have $R/N(R)$ is a ring and $\forall\ r,s \in R$, we have
  \[\left(r+N(R)\right)\left(s+N(R)\right) = rs + N(R).\]
  Let $r + N(R) \in N(R/N(R))$, then $\exists \ n \in \bbn$ such that 
  \[\left(r + N(R)\right)^n = N(R).\]
  Then
  \[r^n +  N(R) = N(R).\]
  So
  \[r^n \in N(R).\]
  Therefore, $\exists \ m \in \bbn$ such that 
  \[(r^n)^m = 0.\]
  Since $R$ is CRW1, we have
  \[(r^n)^m = r^{mn}= 0.\]
  As a result,
  \[r\in N(R).\]
  Thus, $N(R/N(R))=0$.
\end{proof}
\end{exer}

\begin{exer}[Division Algorithm]
Let $R$ be a commutative ring with identity.
Let $f,g\in R[x]$ such that $f$ is monic.
Prove that there exist unique polynomials $q,r\in R[x]$ such that
$g=qf+r$ and $\deg(r)<\deg(f)$.
\\
Hint: For existence, argue by induction on $\deg(g)$. In the induction step, reduce the degree of $g$ by subtracting an appropriate multiple of $f$ to clear the top degree term.
\end{exer}

\begin{proof}	
  First we show the existence.\\
  \textbf{Basic step:}\\ Let $\deg(g) = 0$.\\
  Since $\deg(f) > \deg(r)$, we have $f = 1_R \in R[x]$ which is monic and $r = 0_R \in R[x]$ with $\deg(r) = -\infty$.\\
  Let $q = g \in R[x]$, then we have $g = qf + r$.\\
  \textbf{Inductive step:}\\ Assume when $\deg(g) =n$, there exists $q,r \in R[x]$ such that $g = qf+r$.\\
  Let $g = \sum_{i=0}^{n+1}a_ix^i \in R[x]$ with $a_{n+1} \neq 0_R$.\\
  Let $h = \sum_{i=1}^{n+1}a_ix^{i-1}$, then $\deg(h) = n $. \\
  By assumption, there exists $q,r \in R[x]$ such that $h=qf+r$.\\
  Then
  \begin{align*}
  	g &= hx + a_0\\
  	  &= (qf+r)x+ a_0 \\
	  &=qxf + (rx + a_0). \\ 
  \end{align*}
  \begin{enumerate}[(a)]
  	  \item
  When $\deg(r)+1 < \deg(f)$, we have $rx + a_0 \in R[x]$ and $\deg(rx + a_0) < \deg(f)$.	\\
	Then after letting $q_1 = qx\in R[x]$ and $r_1 = rx + a_0\in R[x]$, we have $g = q_1f + r_1$.\\
	\item
  	When $\deg(r) + 1 = \deg(f)$, assume $\deg(r) =k$ and $r = \sum_{j=1}^kr_ix^i\in R[x] $ with $r_k \neq 0_R$.\\
  	Then $\deg(f) = k+1$ and 
  \begin{align*}
  	 g &=qxf + rx+a_0. \\
  	   &=qxf + \sum_{j=1}^kr_ix^{i+1} + a_0\\
	   	 &=(r_k+qx)f + \left(r_{k}\left(x^{k+1}-f\right) + \sum_{j=1}^{k-1}r_ix^{i+1} + a_0\right).
  \end{align*}
  \end{enumerate}
  Since $\deg(f) =k +1$ and $f$ is monic, we have $\deg\left(x^{k+1}-f\right) \leq k$.\\
  So after letting $r_2 = r_{k}\left(x^{k+1}-f\right) + \sum_{j=1}^{k-1}r_ix^{i+1} + a_0 \in R[x]$,\\
  we have
  \[\deg(r_2) \leq k < \deg(f).\]
  Let $q_2 = r_k+qx \in R[x]$, then we have
 \[g = q_2f+r_2.\]
 Therefore, our assumption also holds when $\deg(g) = n+1$.\\
 As a result, when $\deg(g) =n$, there exists $q,r \in R[x]$ such that $g = qf+r$.\\
 Next, we show the uniqueness.\\ 
 Suppose for $g,f \in R[x]$ and $f$ is monic, $\exists\ q_1,q_2,r_1,r_2$ such that $g = q_1f+r_1 = q_2f+r_2$ with $\deg(r_1)<\deg(f)$ and $\deg(r_2) < \deg(f)$.\\
 Then 
 \[(q_1-q_2)f = r_2-r_1.\]
So
\begin{equation}\label{eq:1}
	\deg(f) \leq \deg(r_2-r_1) \leq \max\{\deg(r_1),\deg(r_2)\}.
\end{equation}
Since $\deg(r_1)<\deg(f)$ and $\deg(r_2) < \deg(f)$, we have
\[\max\{\deg(r_1),\deg(r_2)\} < \deg(f),\]   
  which is contradicted by (\ref{eq:1}).\\
  Thus, let $g,f\in R[x]$ and $f$ is monic, there exist unique $q,r\in R[x]$ such that $g = qf+r$.
\end{proof}  

\end{document}
