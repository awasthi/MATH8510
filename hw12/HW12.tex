\documentclass{amsart}

% PACKAGES

\usepackage{amsmath}
\usepackage{amsfonts}
\usepackage{amssymb,enumerate}
\usepackage{amsthm,stmaryrd}
\usepackage[all]{xy}
\usepackage{hyperref}

%\theoremstyle{definition}
%\newtheorem{exer}{Exercise}

\newcommand{\bbr}{\mathbb{R}}
\newcommand{\bbc}{\mathbb{C}}
\newcommand{\bbz}{\mathbb{Z}}
\newcommand{\bbq}{\mathbb{Q}}
\newcommand{\bbn}{\mathbb{N}}
\newcommand{\be}{\mathbf{e}}
\newcommand{\ba}{\mathbf{a}}
\newcommand{\fm}{\mathfrak{m}}
\newcommand{\Hom}{\operatorname{Hom}}
\renewcommand{\ker}{\operatorname{Ker}}
\newcommand{\im}{\operatorname{Im}}
\newcommand{\xra}{\xrightarrow}
\newcommand{\wti}{\widetilde}

\theoremstyle{plain}
\newtheorem{lem}{Lemma}
\newtheorem{cor}[lem]{Corollary}
\newtheorem{prop}[lem]{Proposition}
\newtheorem{thm}[lem]{Theorem}
\newtheorem{conj}[lem]{Conjecture}
\newtheorem{intthm}{Theorem}
\renewcommand{\theintthm}{\Alph{intthm}}

\theoremstyle{definition}
\newtheorem{defn}[lem]{Definition}
\newtheorem{ex}[lem]{Example}
\newtheorem{question}[lem]{Question}
\newtheorem{questions}[lem]{Questions}
\newtheorem{problem}[lem]{Problem}
\newtheorem{disc}[lem]{Remark}
\newtheorem{rmk}[lem]{Remark}
\newtheorem{construction}[lem]{Construction}
\newtheorem{notn}[lem]{Notation}
\newtheorem{fact}[lem]{Fact}
\newtheorem{para}[lem]{}
\newtheorem{exer}[lem]{Exercise}
\newtheorem{remarkdefinition}[lem]{Remark/Definition}
\newtheorem{notation}[lem]{Notation}
\newtheorem{step}{Step}
\newtheorem{convention}[lem]{Convention}
\newtheorem*{Convention}{Convention}
\newtheorem{assumption}[lem]{Assumption}

\newcommand{\fmn}{F^{m\times n}}
\newcommand{\fnn}{F^{n\times n}}
\newcommand{\col}{\operatorname{Col}}
\newcommand{\row}{\operatorname{Row}}
\newcommand{\Span}{\operatorname{Span}}	
\newcommand{\rank}{\operatorname{rank}}	
\newcommand{\OO}[1]{\mathcal{O}_{#1}}
\begin{document}

\noindent MATH 8510, Abstract Algebra I \\
Fall 2016\\
Exercises 12\\
Collaborators: Daozhou Zhu,  Xiaoyuan Liu\\
Name: \textbf{Shuai Wei}

\

%\noindent
%Throughout this homework set, let $F$ be a field
%
%
%\

\begin{exer}
Let $R$ be a commutative ring with identity.
Let $f\in R[x]$ be monic of degree $d\geq 1$.
In the quotient ring $S=R[x]/fR[x]$, for all $g\in R[X]$, set $\overline g:=g+fR[x]$.
\begin{enumerate}[(a)]
\item 
Prove that for every element $s\in S$ there exist unique elements $r_0,\ldots,r_{d-1}\in R$
such that $s=\sum_{k=0}^{d-1}\overline{r_k}\ \overline{x}^k$.
(Hint: Division Algorithm)
\begin{proof}
  Let $s \in S$.\\
  Let $g = f(\overline{x}) \in R[\overline{x}]$.\\
  Since $f \in R[x]$ is monic of degree d, $g$ is monic of degree d.\\
  Then by the Division Algorithm, there is unique $q,r \in S$ such that $s = qg + r$ with $\deg(r) < \deg(f) = d$.\\
  Since $S= R[x]/fR[x]$ and $s \in S$, $\deg(s) < \deg(f) =\deg(g) = d$,\\
  we have 
  \[q= 0_S \in S.\]
 So ther is a unique $r \in S$ with $\deg(r) < d$ such that $s= r$.\\
Namely, there exsits unique elements $\overline{r^\prime_0},\ldots,\overline{r\prime_{d-1}} \in S$ such that $s=\sum_{k=0}^{d-1}\overline{r\prime_k} ~\overline{x}^k$.\\
 Since for $k=0,1,\ldots,d-1$, 
 \[\deg(\overline{r\prime_k}) = 0\] 
 and 
 \[\overline{r\prime_k} = r\prime_k + fR[x],\]
where $r\prime_k \in R[x]$,\\
 in addtion, $\deg(f) = d \geq 1$, \\
 we have for $k=0,1,\ldots,d-1$, there is only one representative for $\overline{r\prime_k}$.\\
 Moreover, by the Division Algorithm, given $r\prime_k \in R[x]$, there exists unique $r_k, p \in R[x]$ such that 
 \[r\prime_k = pf + r_k.\]
 Then we have $r_k = r\prime_k +f(-p) \in \overline{r\prime_k}$ is the unique representative for $\overline{r\prime_k}$ for $k=0,1,\ldots,d-1$. \\
 Then
 \[\deg(r_k) = \deg(\overline{r\prime_k}) = 0.\]
So such $r_k \in R$ is unique for $k=0,1,\ldots,d-
1$.\\
Therefore, for every element $s\in S$, there exists unique elements $r_0,\ldots,r_{d-1}\in R$ such that $s=\sum_{k=0}^{d-1}\overline{r_k}\ \overline{x}^k$.
\end{proof}
\item Prove that the function $\epsilon\colon R\to S$ given by $r\mapsto\overline r$ is a ring monomorphism, that is, a 1-1 ring homomorphism.
Conclude that $R\cong\im(\epsilon)\subseteq S$.\\
Note: We often use this to identify $R$ with its image in $S$, so that, e.g., we think of $R$ as a subring of $S$, and the formula in part~(a) becomes $s=\sum_{k=0}^{d-1}r_k \overline{x}^k$.
\begin{proof}
  Since $\forall \ r,s \in R$, we have $r,s\in R[x]$, and since $fR[x] \leq R[x]$, we have
	\begin{align*}
	  \epsilon(r+s) &=\overline{r+s} \\ 
	  				&=r+s+fR[x]\\
	  				&=(r+fR[x]) + (s+fR[x]) \\
	  				&=\overline{r} + \overline{s} \\
	  				&=\epsilon(r)+\epsilon(s),
	\end{align*}
	and 
	\begin{align*}
	  \epsilon(rs) &=\overline{rs} \\ 
	  				&=rs+fR[x]\\
	  				&=(r+fR[x])(s+fR[x]) \\
	  				&=\overline{r}~ \overline{s} \\
	  				&=\epsilon(r)\epsilon(s),
	\end{align*}
	it is a ring homomorphism.\\
	Let $r, s \in R$, if $\overline{r} = \overline{s}$, then $r=s$ since $\overline{r}$ and $\overline{s}$ have unique polynomial form, respectively, by part (a).\\
	Let $r \in R$, then  
	\begin{align*}
	  r \in \ker(\epsilon) &\Longleftrightarrow \epsilon(r) = 0_s \\
&\Longleftrightarrow \overline{r} = \overline{0_r} \\
&\Longleftrightarrow r = 0_r. 
	\end{align*}
	So $\epsilon$ is 1-1.\\
	Thus, $\epsilon$ is a ring monomorphism.\\
	Since $\im(\epsilon) \subseteq S$ and it is onto from $R$ to $S$, we have
	\[R \cong \im(\epsilon) \subseteq S. \]
\end{proof}
\item What happens in part~(b) when $d=0$? Specifically, what can you say about $S$ and $\epsilon$ in this case?\\
  \textbf{Soln:} In the proof of part (a), we know when $d = 0$, for $s \in S$, $s$ may have more than one representative from $R$, so we can not make conclusion for part (a).\\
Also, since for $s \in S$, $s$ may have alternative polynomial form, $\epsilon$ may not be 1-1, but is still a homomorphism.\\
At last, $\im(\epsilon) \subseteq S$ always holds.

\end{enumerate}
\end{exer}

\begin{exer}[$\exists\bbc$]
Consider the ring $C=\bbr[x]/(x^2+1)\bbr[x]$, and set $i:=\overline x\in C$.
\begin{enumerate}[(a)]
\item 
Prove that 
$i^2=-1$.
\begin{proof}
  \begin{align*}
  	i^2+\overline{1} &=\overline{x}^2+\overline{1}\\
&= \overline{x}~\overline{x}+\overline{1}\\
&=\overline{x^2}+\overline{1}\\
&=\overline{x^2+1}\\
&=\overline{0},
  \end{align*}
  so by Exercise 12$\#$2(b),
  \[i^2 = \overline{0} - \overline{1} = \overline{-1} = -1.\]
\end{proof}
\item 
Using Exercise~1(b) identify $\bbr$ with its image in $C$.
%\item 
Observe that Exercise~1(a) implies that for every element $z\in C$ there exist unique elements $a,b\in \bbr$
such that $z=a+bi$. (There is nothing to prove here.)\\
\textbf{Soln:}\\
Since $\deg(x^2 +1)  = 2 \geq 1$, by Exercise 12$\#$1(a), we have for $c \in C$, there exists unique $a,b \in \bbr$ such that 
\[c = a+bi.\]
Let $r \in \bbr$, then by Exercise 12$\#$1(b),
  \[r = \overline{r}  = r + 0i.\]
\item 
Prove that $(a+bi)+(c+di)=(a+c)+(b+d)i$ and $(a+bi)(c+di)=(ac-bd)+(ad+bc)i$.
\begin{proof}
  Since $a, bi,c,di \in R[\overline{x}]$ and $R[\overline{x}]$ is a polynomial ring, we have the commutative law and associated law of addition inherit from the polynomial ring $R[\overline{x}]$.\\
	So 
  \begin{align*}
  	(a+bi)+(c+di) &= a+c+bi+di\\
  				  &=(a+c)+(b+d)i. 
  \end{align*}
  Since $\bbr$ is a commutative ring, $R[x]$ is also commutative ring, so are $C$ and $R[\overline{x}]$.\\
  Since $a, bi,c,di \in R[\overline{x}]$, we have the commutative law and associated law of addition and multiplication and the distributative law inherit from the polynomial ring $R[\overline{x}]$.\\
  So 
  \begin{align*}
  	(a+bi)(c+di) &=ac + adi +bci+dbi^2 \\
  				 &=ac + adi +bci-db\\
  				 &=(ac-bd)+(ad+bd)i.
  \end{align*}
	

\end{proof}
%\item 
\end{enumerate}
Note that this can be used to show that $C$ satisfies the properties defining the field of complex numbers.
In particular, the ideal $(x^2+1)\bbr[x]$ is maximal.
(You are not required to prove anything here.)
\end{exer}




\end{document}
