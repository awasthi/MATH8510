\documentclass{amsart}

% PACKAGES

\usepackage{amsmath}
\usepackage{amsfonts}
\usepackage{amssymb,enumerate}
\usepackage{amsthm,stmaryrd}
\usepackage[all]{xy}
\usepackage{hyperref}

%\theoremstyle{definition}
%\newtheorem{exer}{Exercise}

\newcommand{\bbr}{\mathbb{R}}
\newcommand{\bbc}{\mathbb{C}}
\newcommand{\bbz}{\mathbb{Z}}
\newcommand{\bbq}{\mathbb{Q}}
\newcommand{\bbn}{\mathbb{N}}
\newcommand{\be}{\mathbf{e}}
\newcommand{\ba}{\mathbf{a}}
\newcommand{\fm}{\mathfrak{m}}
\newcommand{\Hom}{\operatorname{Hom}}
\renewcommand{\ker}{\operatorname{Ker}}
\newcommand{\im}{\operatorname{Im}}
\newcommand{\xra}{\xrightarrow}
\newcommand{\wti}{\widetilde}

\theoremstyle{plain}
\newtheorem{lem}{Lemma}
\newtheorem{cor}[lem]{Corollary}
\newtheorem{prop}[lem]{Proposition}
\newtheorem{thm}[lem]{Theorem}
\newtheorem{conj}[lem]{Conjecture}
\newtheorem{intthm}{Theorem}
\renewcommand{\theintthm}{\Alph{intthm}}

\theoremstyle{definition}
\newtheorem{defn}[lem]{Definition}
\newtheorem{ex}[lem]{Example}
\newtheorem{question}[lem]{Question}
\newtheorem{questions}[lem]{Questions}
\newtheorem{problem}[lem]{Problem}
\newtheorem{disc}[lem]{Remark}
\newtheorem{rmk}[lem]{Remark}
\newtheorem{construction}[lem]{Construction}
\newtheorem{notn}[lem]{Notation}
\newtheorem{fact}[lem]{Fact}
\newtheorem{para}[lem]{}
\newtheorem{exer}[lem]{Exercise}
\newtheorem{remarkdefinition}[lem]{Remark/Definition}
\newtheorem{notation}[lem]{Notation}
\newtheorem{step}{Step}
\newtheorem{convention}[lem]{Convention}
\newtheorem*{Convention}{Convention}
\newtheorem{assumption}[lem]{Assumption}

\newcommand{\fmn}{F^{m\times n}}
\newcommand{\fnn}{F^{n\times n}}
\newcommand{\col}{\operatorname{Col}}
\newcommand{\row}{\operatorname{Row}}
\newcommand{\Span}{\operatorname{Span}}	
\newcommand{\rank}{\operatorname{rank}}	
\newcommand{\OO}[1]{\mathcal{O}_{#1}}
\begin{document}

\noindent MATH 8510, Abstract Algebra I \\
Fall 2016\\
Exercises 9-2\\
Due date Thu 27 Oct 4:00PM

\

%\noindent
%Throughout this homework set, let $F$ be a field
%
%
%\

\begin{exer}[6.1.1]
Let $G$ be a group. 
Prove that $Z_i(G)$ is a characteristic subgroup of $G$ for all $i$. 
\begin{proof}
	We will show it by induction.\\
	\textbf{ Basic steps:}\\
	Let $\sigma \in \operatorname{Aut}(G)$.\\
	$Z_0(G) =\{e_G\}$.\\
	Since $\sigma \in \operatorname{Aut}(G)$, $\sigma(e_G) = e_G$. \\
	So
	\[\sigma(Z_0(G)) = \sigma(Z_0(G)).\]
	Thus, $Z_0(G)$ is a characteristic subgroup of $G$.\\
	$Z_1(G) = Z(G)$. \\
	Since $\sigma \in \operatorname{Aut}(G)$, $\sigma(g^{-1})  = (\sigma(g))^{-1} \in G, \forall\ g \in G$.\\
	Let $z \in Z(G)$\\
	Then $z\sigma(g^{-1}) = \sigma(g^{-1})z, \forall\ g \in G$.\\
    Then
    \[\sigma(z\sigma(g^{-1})) = \sigma(\sigma(g^{-1})z), \forall\ g \in G\]
    Since $\sigma \in \operatorname{Aut}(G)$, we have 
    \[\sigma(z)\sigma\left(\sigma(g^{-1})\right) = \sigma\left(\sigma(g^{-1})\right)\sigma(z), \forall\ g \in G.\]
    Namely,
    \[\sigma(z)g = g\sigma(z), \forall\ g \in G.\]
    Then 
    \[\sigma(z) \in Z(G).\]
    So
	\[\sigma(Z(G)) \subset Z(G).\]
	By the Theorem 4.4.8, $Z(G)$ is a characteristic subgroup of $G$.\\
	\textbf{Induction steps: }\\
	Assume $Z_i(G)$ is a characteristic subgroup of $G$.\\
	Let $\sigma \in \operatorname{Aut}(G)$.\\
	Then $\sigma(Z_i(G)) = Z_i(G)$.\\
	Define 
	\begin{align*}
	  \overline{\sigma} : G/Z_i(G) &\to G/Z_i(G) \\
	  						gZ_i(G) &\to \sigma(g)Z_i(G)  
	\end{align*}
	Then $\overline{\sigma}$ is well-defined since $Z_i(G) \unlhd G$ and $\sigma:G\to G$ is a isomomorphism.\\
	Next we show $\overline{\sigma}$ is an automomorphism.\\
	Let $gZ_i(G), hZ_i(G) \in G/Z_i(G)$, where $g,h\in G$.\\
	Since $Z_i(G) \unlhd G$,
	\begin{align*}
	  \overline{\sigma}(gZ_i(G)hZ_i(G)) &= \overline{\sigma}(ghZ_i(G))\\
	  								 	&= \sigma(gh)Z_i(G)\\
	  									&=\sigma(g)\sigma(h)Z_i(G) \\
	  									&= \sigma(g)Z_i(G)\sigma(h)Z_i(G) \\
	  									&= \overline{\sigma}(gZ_i(G)) \overline{\sigma}(hZ_i(G)).
  	\end{align*}
  	So $\overline{\sigma}$ is a homomorphism.\\
  	Since $\sigma$ is a homomorphism, and $Z_i(G)$ is a characteristic subgroup of $G$.
  	\begin{align*}
  	gZ_i(G) \in \ker\left(\overline{\sigma}\right) &\Leftrightarrow \overline{\sigma}(gZ_i(G)) = Z_i(G) \\
  												   &\Leftrightarrow \sigma(g)Z_i(G) = Z_i(G) \\
  	  											   &\Leftrightarrow \sigma(g) \in Z_i(G) \\
  	  											   &\Leftrightarrow \sigma^{-1}(\sigma(g)) \in Z_i(G) \\
  	  											   &\Leftrightarrow g \in Z_i(G).
  	\end{align*}
	So
	\[\ker(\overline{\sigma}) = Z_i(G).\]
	So $\overline{\sigma}$ is 1-1.\\
	Let $kZ_i(G) \in G/Z_i(G)$, where $k \in G$.\\
	We know $\sigma^{-1}(k) \in G$ since $\sigma^{-1} \in \operatorname{Aut}(G)$.\\
	Then 
	\[\sigma^{-1}(k) Z_i(G) \in G/Z_i(G).\]
	Since 
	\[\overline{\sigma} \left(\sigma^{-1}(k)Z_i(G)\right) = \sigma(\sigma^{-1}(k)) Z_i(G) = kZ_i(G),\]
	$\overline{\sigma}$ is onto.\\
	Thus, 
	\[\overline{\sigma} \in \operatorname{Aut}\left(G/Z_i(G)\right).\]
	Since in basic steps we have shown the center of a group is a characteristic subgroup of the group,\\
	\[\overline{\sigma}\left(Z\left(G/Z_i(G)\right)\right) = Z\left(G/Z_i(G)\right). \]
	By definition, we have 
	\[Z\left(G/Z_i(G)\right)  = Z_{i+1}(G)/Z_i(G).\]
	So
  	\[\overline{\sigma}\left( Z_{i+1}(G)/Z_i(G) \right) = Z_{i+1}(G)/Z_i(G). \]
	Let $z \in Z_{i+1}(G)$, then 
		\[zZ_i(G) \in Z_{i+1}(G)/Z_i(G).\]
	Then
	\[\overline{\sigma}( zZ_i(G)) \in Z_{i+1}(G)/Z_i(G). \]
	Namely,
	\[\sigma(z)Z_i(G) \in  Z_{i+1}(G)/Z_i(G).\]
	So 
  	\[\sigma(z) \in Z_{i+1}(G).\]
	Thus,
  	\[\sigma\left(Z_{i+1}(G)\right) \subset Z_{i+1}(G).\]
	By the Theorem 4.4.8, $Z_{i+1}(G)$ is a characteristic subgroup of $G$.\\
	So the assumption also holds for $Z_{i+1}(G)$.\\
	Thus, we conclude that $Z_i(G)$ is a characteristic subgroup of $G$ for each $i$.
 \end{proof}
\end{exer}

\begin{exer}[6.1.6]
Let $G$ be a group. 
Prove that $G/Z(G)$ is nilpotent if and only if $G$ is nilpotent.
\end{exer}
\begin{proof}
  We first find the relationship between $({G/Z(G)})^n$ and $G^n$.\\
    $(G/Z(G))^0 = (G/Z(G)$.\\
	 Since $Z(G) \unlhd G$,
	\begin{align*}
  	  ({G/Z(G)})^1 &= [G/Z(G),(G/Z(G))^0] \\ 
  	  			 &= [G/Z(G),G/Z(G)] \\
  	  		 	 &= \left\langle [hZ(G), kZ(G)]|hZ(G),kZ(G) \in G/Z(G)\right\rangle \\
  	  			 &= \langle (hZ(G))^{-1}(kZ(G))^{-1}(hZ(G))(kZ(G)) | hZ(G),kZ(G) \in G/Z(G) \rangle\\
  	  			 &=\langle (h^{-1}Z(G))(k^{-1}Z(G))(hZ(G))(kZ(G)) |  hZ(G),kZ(G) \in G/Z(G) \rangle \\
  	  			 &=\langle h^{-1}k^{-1}hkZ(G) | h,k \in G \rangle.\\
  	  			 &=[G,G]/Z(G)\\
  	  			 &=G^1 / Z(G) 
	\end{align*}
	Then 
	\begin{align*}
	({G/Z(G)})^2 &=[G/Z(G),({G/Z(G)})^1] \\
	  			 &=[G/Z(G), G^1 Z(G)]\\
				 &= \left\langle [hZ(G), kZ(G)]|hZ(G) \in G/Z(G), kZ(G) \in G^1/ Z(G)\right\rangle \\
  	  			 &=\langle h^{-1}k^{-1}hkZ(G) | h \in G, k \in G^1 \rangle.\\
  	  			 &=[G,G^1]/Z(G).\\
  	  			 &=G^2/Z(G).
	\end{align*}
	We guess $({G/Z(G)})^n = G^n/Z(G)$.\\
	We will show it by induction.\\
	We have shown the basic steps.\\
	\textbf{Induction steps: }\\
	Assume $({G/Z(G)})^n = G^n/Z(G)$.
	\begin{align*}
	  ({G/Z(G)})^{n+1} &=[G/Z(G),({G/Z(G)})^n] \\
	  			 &=[G/Z(G), G^n/Z(G)]\\
				 &= \left\langle [hZ(G), kZ(G)]|hZ(G) \in G/Z(G), kZ(G) \in G^n /Z(G)\right\rangle \\
  	  			 &=\langle h^{-1}k^{-1}hkZ(G) | h \in G, k \in G^n \rangle.\\
  	  			 &=[G,G^n]/Z(G).\\
  	  			 &=G^{n+1}/Z(G).
	\end{align*}
	So the assumption holds for the $n+1$ case.\\
	Thus, for $n \in \bbn$,
	\[({G/Z(G)})^n = G^n/Z(G).\]
	"$\Leftarrow$". Assume $G$ is nilpotent.\\
	Then $G^m = e_G$ for some $m \geq 0$.\\
	So
	\[({G/Z(G)})^m = G^m/Z(G) = Z(G).\]
	So $G/Z(G)$ is nilpotent.\\
	"$\Rightarrow$". Assume $G/Z(G)$ is nilpotent.\\
	Then $(G/Z(G))^n = Z_G$ for some $n \geq 0$.\\
	So
	\[G^n/Z(G) = ({G/Z(G)})^n = Z(G).\]
	As a result, we have 
	\[G^n \subset Z(G).\]
	Therefore,
	\begin{align*}
	 	G^{n+1} &= [G,G^n] \\
				&=\left\langle [h, k]|h \in G, k \in G^n \right\rangle \\
				&=\left\langle h^{-1}k^{-1}hk|h \in G, k \in G^n \right\rangle \\
				&=\left\langle h^{-1}k^{-1}kh|h \in G, k \in G^n \right\rangle \\
				&=\left\langle e_G|h \in G, k \in G^n \right\rangle \\
	  			&=\langle e_G \rangle \\
	  			&= e_G,
  	\end{align*}
  	since $G^n \subset Z(G)$.\\
	Thus, $G$ is nilpotent.
 \end{proof}

\end{document}



