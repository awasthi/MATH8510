\documentclass{amsart}

% PACKAGES

\usepackage{amsmath}
\usepackage{amsfonts}
\usepackage{amssymb,enumerate}
\usepackage{amsthm,stmaryrd}
\usepackage[all]{xy}
\usepackage{hyperref}

%\theoremstyle{definition}
%\newtheorem{exer}{Exercise}

\newcommand{\bbr}{\mathbb{R}}
\newcommand{\bbc}{\mathbb{C}}
\newcommand{\bbz}{\mathbb{Z}}
\newcommand{\bbq}{\mathbb{Q}}
\newcommand{\bbn}{\mathbb{N}}
\newcommand{\be}{\mathbf{e}}
\newcommand{\ba}{\mathbf{a}}
\newcommand{\fm}{\mathfrak{m}}
\newcommand{\Hom}{\operatorname{Hom}}
\renewcommand{\ker}{\operatorname{Ker}}
\newcommand{\im}{\operatorname{Im}}
\newcommand{\xra}{\xrightarrow}
\newcommand{\wti}{\widetilde}

\theoremstyle{plain}
\newtheorem{lem}{Lemma}
\newtheorem{cor}[lem]{Corollary}
\newtheorem{prop}[lem]{Proposition}
\newtheorem{thm}[lem]{Theorem}
\newtheorem{conj}[lem]{Conjecture}
\newtheorem{intthm}{Theorem}
\renewcommand{\theintthm}{\Alph{intthm}}

\theoremstyle{definition}
\newtheorem{defn}[lem]{Definition}
\newtheorem{ex}[lem]{Example}
\newtheorem{question}[lem]{Question}
\newtheorem{questions}[lem]{Questions}
\newtheorem{problem}[lem]{Problem}
\newtheorem{disc}[lem]{Remark}
\newtheorem{rmk}[lem]{Remark}
\newtheorem{construction}[lem]{Construction}
\newtheorem{notn}[lem]{Notation}
\newtheorem{fact}[lem]{Fact}
\newtheorem{para}[lem]{}
\newtheorem{exer}[lem]{Exercise}
\newtheorem{remarkdefinition}[lem]{Remark/Definition}
\newtheorem{notation}[lem]{Notation}
\newtheorem{step}{Step}
\newtheorem{convention}[lem]{Convention}
\newtheorem*{Convention}{Convention}
\newtheorem{assumption}[lem]{Assumption}

\newcommand{\fmn}{F^{m\times n}}
\newcommand{\fnn}{F^{n\times n}}
\newcommand{\col}{\operatorname{Col}}
\newcommand{\row}{\operatorname{Row}}
\newcommand{\Span}{\operatorname{Span}}	
\newcommand{\rank}{\operatorname{rank}}	
\newcommand{\OO}[1]{\mathcal{O}_{#1}}
\begin{document}

\noindent MATH 8510, Abstract Algebra I \\
Fall 2016\\
Exercises 9-1\\
Name: Shuai Wei\\
Collaborator: DaoZhou Zhu, XiaoYuan Liu

\

%\noindent
%Throughout this homework set, let $F$ be a field
%
%
%\

\begin{exer}[5.5.1--2]
Let $H$ and $K$ be groups, let $\varphi\colon K\to \operatorname{Aut}(H)$ be a homomorphism, and let $\wti H,\wti K$ be as in the Theorem~5.5.3.
Prove that $C_{\wti K}(\wti H)\cong\ker(\varphi)$
and $C_{\wti H}(\wti K)=N_{\wti H}(\wti K)$.

\begin{proof}
	Define $\phi$ as
	\begin{align*}
	  \phi: \ker(\varphi) & \to C_{\wti K}(\wti H)\\
	  	k &\mapsto (e_H,k)
	\end{align*}
	First we show $\phi$ is well defined.\\
	Let $k \in \ker(\varphi)$, then $k \in \ker(\varphi) \leq K$ by the definition of $\varphi$.\\ 
	$\forall (h,e_K) \in \wti H$, where $h \in H$, by the Theorem 5.5.3 (d), we have 
	\begin{align*}
	  (e_H,k)(h,e_K)(e_H,k)^{-1} &= (\varphi(k)h, e_K).
	\end{align*}
	Since $k \in \ker(\varphi)$, $\varphi(k) = e_{\operatorname{Aut}(H)} $, where $e_{\operatorname{Aut}(H)}$ is the identity map from $H$ to $H$.\\
	  Besides, $h \in H$, so 
	  \[\varphi(k)h  = e_{\operatorname{Aut}(H)}h =h.\]
	  Then 
	  \[(e_H,k)(h,e_K)(e_H,k)^{-1} =(h, e_K) \in \wti H.\]
	  So $(e_H,k) \in C_{\wti K}(\wti H)$.\\
	  Therefore, $\phi$ is well defined. \\
	  Next we show $\phi$ is a homomorphism.\\
	 $\forall\ k_1,k_2 \in \ker(\varphi)$, 
	 \begin{align*}
	   \phi(k_1k_2) &=(e_H,k_1k_2) \\
	   			    &=\left(e_H\cdot\varphi(k_1)(e_H),k_1k_2\right)\\
	   				&=(e_H,k_1)(e_H,k_2)\\
	   				&=\phi(k_1)\phi(k_2),
	 \end{align*}
	 so $\phi$ is a homomorphism.\\
	Let $k \in \ker(\varphi)$.\\
	Then
	\begin{align*}
	  k \in \ker(\phi) &\Leftrightarrow \phi(k) = (e_H,e_K) \\
	 				   &\Leftrightarrow (e_H,k) = (e_H,e_K) \\
	  				   &\Leftrightarrow k = e_K.
	\end{align*}
   So 
   \[\ker(\phi) = \{e_K\} = \{e_{\ker(\varphi)}\}.\]
   Thus, $\phi$ is 1-1.\\
   Let $(e_H,k) \in C_{\wti K}(\wti H)$, where $k \in K$.\\
   Then $\forall\ (h,e_K) \in \wti H$, we have $h\in H$ is arbitrary,\\
   and by the Theorem 5.5.3 (d),
   \begin{align*}
   	 (e_H,k) \in C_{\wti K}(\wti H) &\Leftrightarrow  (e_H,k) (h,e_K)(e_H,k)^{-1}  = (h,e_K), \ \ \forall\ h \in H\\
   	 								&\Leftrightarrow (\varphi(k)h,e_K) = (h,e_K), \ \ \forall\ h \in H\\
   	 								&\Leftrightarrow \varphi(k)h = h, \ \ \forall\ h \in H\\
   	 								&\Leftrightarrow \varphi(k) = e_{\operatorname{Aut}(H)}\\
  									&\Leftrightarrow k \in \ker(\varphi).
   \end{align*}
  	Namely, for any $(e_H,k) \in C_{\wti K}(\wti H)$, we have $\ k \in \ker(\varphi)$ such that $\phi(k) = (e_H,k)$.\\
  	Thus, $\phi$ is onto.\\
  	Next we show 
  	\[C_{\wti H}(\wti K) = N_{\wti H}(\wti K)\]
  	By the definition of $C_{\wti H}(\wti K)$ and $N_{\wti H}(\wti K)$, we have
  	\[C_{\wti H}(\wti K) \subset N_{\wti H}(\wti K).\]
  	Let $(h,e_K) \in N_{\wti H}(\wti K)$, where $h \in H$.\\
  	Let $(e_H,k) \in \wti K$, where $k \in K$.\\ 
  	Since $\varphi$ is a homomorphism and $\varphi(e_K) = e_{\operatorname{Aut}(H)}$,
	\begin{align*}
	   		  (h,e_K) (e_H,k)(h,e_K)^{-1} &=(h\cdot \varphi(e_K)(e_H),e_Kk)(h,e_K)^{-1} \\
			   &=(he_H,k)\left(\varphi(e_K^{-1})(h^{-1}),e_K^{-1}\right)\\
			   &=(h,k)(\varphi(e_K)h^{-1},e_K)\\
	  			&=(h,k)\left(h^{-1},e_K\right)\\
	  		    &=\left(h\cdot\varphi(k)(h^{-1}),ke_K\right)\\
	  			&=\left(h\cdot\varphi(k)(h^{-1}),k\right)\\
	  		  	&\in \wti K.
	\end{align*}
	So
	\[h\cdot\varphi(k)(h^{-1}) = e_H.\]
	Namely, $\forall\ (h,e_K) \in N_{\wti H}(\wti K)$, $\forall\ (e_H,k)\in \wti K$, we have
	\[(h,e_K) (e_H,k)(h,e_K)^{-1} = \left(e_H,k\right).\]
	So
	\[(h,e_K) \in C_{\wti H}(\wti K).\]
	Thus,
	\[N_{\wti H}(\wti K) \subset C_{\wti H}(\wti K).\]
	As a result,
	\[C_{\wti H}(\wti K) = N_{\wti H}(\wti K).\]
\end{proof}


\end{exer}

\begin{exer}[5.5.11]
Classify all groups of order 28. (There are four isomorphism types. Feel free to use Proposition 11 and/or Exercise 6 from this section of the text;
You do not need to solve Exercise 6.)\\
\textbf{soln:} Let $G$ be a group of order 28.\\
If $G$ is abelian, since $28 = 2^2\times7$ and $2=2 = 1+1$, by the FTFGAG, \\
we have 2 types of isomorphic groups of order 28 and they are 
\begin{gather*}
	G \cong \bbz/4\bbz \times \bbz/7\bbz\\
	G \cong \bbz/2\bbz \times \bbz/2\bbz  \times \bbz/7\bbz.
\end{gather*}
The invariant factor decomposition of them are 
\begin{gather*}
	G \cong \bbz/28\bbz \\
	G \cong \bbz/14\bbz \times \bbz/2\bbz.	
\end{gather*}
Next we discuss the case when $G$ is not abelian.\\
By the Sylow theorem $4.5.4$, we have $n_7 \equiv (1 \mod{11})$ and $n_7 | 4$.\\
$n_7 | 4$, so $n_7 =1$ or $2$ or $4$.\\
Besides, $n_7 \equiv (1 \mod{11})$, so $n_7 = 1$.\\
Let $H \in Syl_{7}(G)$, then $|H| = 7$.\\
Then by the Corollary 4.5.5, we have 
\[H \unlhd G.\] 
Let $K \in Syl_2(G)$, then $|K| = 4$.\\
Then $KH \leq G$ by the Corollary 3.2.7.\\
So 
\[HK=KH \leq G\] 
by the Theorem 3.2.6.\\
We claim $H\cap K = \{e_G\}$.\\
Assume $\exists\ x \in G$ and $x \neq e_G$ such that $x\in H\cap K$.\\
Then $\langle x \rangle \leq H$.\\
So $|x| \mid |H|$.\\
Since $|x| \neq 1$ and $|H| = 7$, we have $|x| = 7 > |K|$, which is a contradiction since $x \in K$.\\
Thus, 
\[H\cap K = \{e_G\}.\]
Then by the Theorem 5.5.5, $\exists\ \varphi:K \to \operatorname{Aut}(H)$ such that 
\[G=HK \cong H\ \rtimes_\varphi K. \]
Moreover, let $x \in H$ and $x \neq e_G$, similarly, we have 
\[H = \langle x \rangle.\]
We claim $K$ is abelian.\\
By the Cauchy Theorem, $\exists\ x \in K$ such that $|x| = 2$.\\
Then $x^{-1} = x$.\\
So $\{e_G,x\} \subset G$.
Since $|K|  =4$, $\exists \ y \in K$ such that $y\neq e_G$ and $y\neq x$.\\
If $y^{-1} = x$, then $x = x^{-1} = y$, which is a contradiction.\\
So $y^{-1} \neq x$.\\
Similarly, we have $y^{-1}\neq e_G$ and $y^{-1} \neq y$.\\
Then 
\[K =\{e_G,x,y,y^{-1}\}.\] 
So we have $xy, yx\in K$.\\
Since $x,y,\neq e_G$ and $y^{-1} \neq x$, we have $xy = y^{-1}$.\\
Similarly, we have $yx = y^{-1}$. \\
So 
\[y^{-1}= xy = yx.\]
Similarly, we have 
\[y=xy^{-1} = y^{-1}x .\]
Thus, $K$ is abelian of order 4.\\
Since $4 = 2^2$ and $2=2=1+1$, by the FTFGAG,
\[K \cong Z_4.\]
and \[K \cong Z_2 \times Z_2.\]
Since we consider the types of groups which is isomorphic to non-abelian groups of order 28, we can set $H = \bbz/7\bbz$ and $K = \bbz/4\bbz$ or $K= \bbz/2\bbz \times \bbz/2\bbz$. \\
Since 
\[\operatorname{Aut}(H) \cong (\bbz /7\bbz)^{\times},\]
we have
\[|\operatorname{Aut}(H)| = |(\bbz /7\bbz)^{\times}| = \varphi(7)=6.\]
Since $(\bbz /7\bbz)^{\times}$ is abelian and $6 = 2\times 3$, by the FTFGAG,
\[(\bbz /7\bbz)^{\times} \cong \bbz/6\bbz.\]
Since $\bbz/6\bbz$ is cyclic, we have $\operatorname{Aut}(H)$ is also cyclic.\\
Let $\operatorname{Aut}(H) = \langle \bar{1}_6 \rangle$.
\begin{enumerate}[(a)]
	\item Consider $K=\bbz/4\bbz$.\\
	  Let $K=\bbz/4\bbz = \langle \bar{1}_4 \rangle$.\\
	  Then we consider the homomorphism 
	  \begin{align*}
		\varphi: \langle \bar{1}_4& \rangle \to \langle \bar{1}_6 \rangle 
  	  \end{align*}
	  Since $\bbz/4\bbz = \langle \bar{1}_4 \rangle$ , $\varphi$ is uniquely determined by $\varphi(\bar{1}_4)$.\\
	  Write $\varphi(\bar{1}_4) = \bar{a} \in \langle \bar{1}_6 \rangle$.\\
	  Since $\varphi$ is a homomorphism and $|\bar{1}_4| = 4$,
  	  \[4\varphi(\bar{1}_4) = \varphi(\bar{4}_4) = \varphi(\bar{0}_4) = \bar{0}_6.\]
  	  Then  
  	  \[\varphi(\bar{1}_4) =\overline{3k}_6, \]
  	  where $k=\{0\} \cup \bbn$.\\
	  Since $\langle \bar{1}_6 \rangle = \{\bar{0}_6,\bar{1}_6,\bar{2}_6,\bar{3}_6,\bar{4}_6,\bar{5}_6\}$,
	  \[\varphi(\bar{1}_4) = \bar{0}_6 \text{ or } \bar{3}_6.\]
	  If $\varphi(\bar{1}_4) = \bar{0}_6$, then $\varphi$ is a trival homomorphism and then $G \cong H \rtimes_\varphi K$ becomes 
	  \[G \cong H \times K = \bbz/7\bbz \times \bbz/4\bbz,\]
	  which is a contradiction since $G$ is non-abelian by assumption.\\
	  So 
	  \[\varphi(\bar{1}_4) = \bar{3}_6.\]
	Thus, 
	\[G \cong ( \bbz/7\bbz) \rtimes_\varphi ( \bbz/4\bbz),\]
	with the homomorphism $\varphi$ determined by $\varphi(\bar{1}_4) = \bar{3}_6$.\\
	\item Consider $K = \bbz/2\bbz \times \bbz/2\bbz$.\\
	  Let $K=\langle \bar{1_a} \rangle \times \langle \bar{1_b} \rangle$.\\
	Then $\varphi$ is uniquely determined by $\varphi(\bar{1}_a)$ and $\varphi(\bar{1}_b)$.\\
	Let $\varphi:  \bbz/2\bbz \to \langle \bar{1}_6 \rangle$ determined by $\varphi(\bar{1}_a)$ be a homomorphism. \\
	Write $\varphi(\bar{1}_a) = \bar{x} \in \langle \bar{1}_6 \rangle $.\\
	  Since $\varphi$ is a homomorphism and $|\bar{1}_a| = 2$,
  	  \[2\varphi(\bar{1}_a) = \varphi(\bar{2}_a) = \varphi(\bar{0}_a) = \bar{0}_6.\]
  	  Then  
  	  \[\varphi(\bar{1}_a) =\overline{3k}_6, \]
  	  where $k=\{0\} \cup \bbn$.\\
	  Since $\langle \bar{1}_6 \rangle = \{\bar{0}_6,\bar{1}_6,\bar{2}_6,\bar{3}_6,\bar{4}_6,\bar{5}_6\}$,
	  \[\varphi(\bar{1}_a) = \bar{0}_6 \text{ or } \bar{3}_6.\]
	  Let $\varphi:  \bbz/2\bbz \to \langle \bar{1}_6 \rangle$ determined by $\varphi(\bar{1}_b)$ be a homomorphism.\\
	  Similarly, we have
		\[\varphi(\bar{1}_b) = \bar{0}_6 \text{ or } \bar{3}_6.\]
	  \begin{enumerate}[(i)]
		\item
		  If $\varphi(\bar{1}_a) = \varphi(\bar{1}_b) = \bar{0}_6$, then $\phi$ is trivial, similarly, we can find this is contradicted by that $G$ is non-abelian.\\
		\item
		  If $\varphi(\bar{1}_a) = \bar{0}_6$ and $\varphi(\bar{1}_b) = \bar{3}_6$, then 
		  \[G \cong \bbz/7\bbz \rtimes_\varphi (\bbz/2\bbz \times \bbz/2\bbz). \]
		\item
		  If $\varphi(\bar{1}_b) = \bar{0}_6$ and $\varphi(\bar{1}_a) = \bar{3}_6$,\\ 
		  let $\varphi$ determined by $\varphi(\bar{1}_a) = \bar{0}_6$ and $\varphi(\bar{1}_b) = \bar{3}_6$ be $\varphi_1$.\\
		  let $\varphi$ determined by $\varphi(\bar{1}_b) = \bar{0}_6$ and $\varphi(\bar{1}_a) = \bar{3}_6$ be $\varphi_2$.\\
		  By Symmetry, we have 
		  \[\varphi_1(\bbz/2\bbz \times \bbz/2\bbz) = \varphi_2(\bbz/2\bbz \times \bbz/2\bbz).\]
		  Since $\bbz/2\bbz \times \bbz/2\bbz$ is cyclic, by Exercise 6 from this section of the text,\\
		  we have 
		  \[\bbz/7\bbz \rtimes_{\varphi_1} (\bbz/2\bbz \times \bbz/2\bbz) \cong \bbz/7\bbz \rtimes_{\varphi_2} (\bbz/2\bbz \times \bbz/2\bbz)\]
		 \item
		  If $\varphi(\bar{1}_a) = \bar{3}_6$ and $\varphi(\bar{1}_b) = \bar{3}_6$, it seems no new homomorphism produced.\\ 
	\end{enumerate}
	In summary, we have 4 types of isomorphisms, and they are
	\begin{enumerate}[(a)]
	\item
	$\bbz/28\bbz$.
	\item
	$\bbz/14\bbz \times \bbz/2\bbz$.
	\item
	$\bbz/7\bbz) \rtimes_\varphi ( \bbz/4\bbz)$ with the homomorphism $\varphi$ determined by $\varphi(\bar{1}_4) = \bar{3}_6$.
	\item
	$\bbz/7\bbz \rtimes_\varphi (\bbz/2\bbz \times \bbz/2\bbz)$ with the homomorphism $\varphi$ determined by $\varphi(\bar{1}_a) = \bar{0}_6$ and $\varphi(\bar{1}_b) = \bar{3}_6$. 

	\end{enumerate}

\end{enumerate}


\end{exer}


\begin{exer}[6.1.1]
Let $G$ be a group. 
Prove that $Z_i(G)$ is a characteristic subgroup of $G$ for all $i$. 
\begin{proof}
	We will show it by induction.\\
	\textbf{ Basic steps:}\\
	Let $\sigma \in \operatorname{Aut}(G)$.\\
	$Z_0(G) =\{e_G\}$.\\
	Since $\sigma \in \operatorname{Aut}(G)$, $\sigma(e_G) = e_G$. \\
	So
	\[\sigma(Z_0(G)) = \sigma(Z_0(G)).\]
	Thus, $Z_0(G)$ is a characteristic subgroup of $G$.\\
	$Z_1(G) = Z(G)$. \\
	Since $\sigma \in \operatorname{Aut}(G)$, $\sigma(g^{-1})  = (\sigma(g))^{-1} \in G, \forall\ g \in G$.\\
	Let $z \in Z(G)$\\
	Then $z\sigma(g^{-1}) = \sigma(g^{-1})z, \forall\ g \in G$.\\
    Then
    \[\sigma(z\sigma(g^{-1})) = \sigma(\sigma(g^{-1})z), \forall\ g \in G\]
    Since $\sigma \in \operatorname{Aut}(G)$, we have 
    \[\sigma(z)\sigma\left(\sigma(g^{-1})\right) = \sigma\left(\sigma(g^{-1})\right)\sigma(z), \forall\ g \in G.\]
    Namely,
    \[\sigma(z)g = g\sigma(z), \forall\ g \in G.\]
    Then 
    \[\sigma(z) \in Z(G).\]
    So
	\[\sigma(Z(G)) \subset Z(G).\]
	By the Theorem 4.4.8, $Z(G)$ is a characteristic subgroup of $G$.\\
	\textbf{Induction steps: }\\
	Assume $Z_i(G)$ is a characteristic subgroup of $G$.\\
	Let $\sigma \in \operatorname{Aut}(G)$.\\
	Then $\sigma(Z_i(G)) = Z_i(G)$.\\
	Define 
	\begin{align*}
	  \overline{\sigma} : G/Z_i(G) &\to G/Z_i(G) \\
	  						gZ_i(G) &\to \sigma(g)Z_i(G)  
	\end{align*}
	Then $\overline{\sigma}$ is well-defined since $Z_i(G) \unlhd G$ and $\sigma:G\to G$ is a isomomorphism.\\
	Next we show $\overline{\sigma}$ is an automomorphism.\\
	Let $gZ_i(G), hZ_i(G) \in G/Z_i(G)$, where $g,h\in G$.\\
	Since $Z_i(G) \unlhd G$,
	\begin{align*}
	  \overline{\sigma}(gZ_i(G)hZ_i(G)) &= \overline{\sigma}(ghZ_i(G))\\
	  								 	&= \sigma(gh)Z_i(G)\\
	  									&=\sigma(g)\sigma(h)Z_i(G) \\
	  									&= \sigma(g)Z_i(G)\sigma(h)Z_i(G) \\
	  									&= \overline{\sigma}(gZ_i(G)) \overline{\sigma}(hZ_i(G)).
  	\end{align*}
  	So $\overline{\sigma}$ is a homomorphism.\\
  	Since $\sigma$ is a homomorphism, and $Z_i(G)$ is a characteristic subgroup of $G$.
  	\begin{align*}
  	gZ_i(G) \in \ker\left(\overline{\sigma}\right) &\Leftrightarrow \overline{\sigma}(gZ_i(G)) = Z_i(G) \\
  												   &\Leftrightarrow \sigma(g)Z_i(G) = Z_i(G) \\
  	  											   &\Leftrightarrow \sigma(g) \in Z_i(G) \\
  	  											   &\Leftrightarrow \sigma^{-1}(\sigma(g)) \in Z_i(G) \\
  	  											   &\Leftrightarrow g \in Z_i(G).
  	\end{align*}
	So
	\[\ker(\overline{\sigma}) = Z_i(G).\]
	So $\overline{\sigma}$ is 1-1.\\
	Let $kZ_i(G) \in G/Z_i(G)$, where $k \in G$.\\
	We know $\sigma^{-1}(k) \in G$ since $\sigma^{-1} \in \operatorname{Aut}(G)$.\\
	Then 
	\[\sigma^{-1}(k) Z_i(G) \in G/Z_i(G).\]
	Since 
	\[\overline{\sigma} \left(\sigma^{-1}(k)Z_i(G)\right) = \sigma(\sigma^{-1}(k)) Z_i(G) = kZ_i(G),\]
	$\overline{\sigma}$ is onto.\\
	Thus, 
	\[\overline{\sigma} \in \operatorname{Aut}\left(G/Z_i(G)\right).\]
	Since in basic steps we have shown the center of a group is a characteristic subgroup of the group,\\
	\[\overline{\sigma}\left(Z\left(G/Z_i(G)\right)\right) = Z\left(G/Z_i(G)\right). \]
	By definition, we have 
	\[Z\left(G/Z_i(G)\right)  = Z_{i+1}(G)/Z_i(G).\]
	So
  	\[\overline{\sigma}\left( Z_{i+1}(G)/Z_i(G) \right) = Z_{i+1}(G)/Z_i(G). \]
	Let $z \in Z_{i+1}(G)$, then 
		\[zZ_i(G) \in Z_{i+1}(G)/Z_i(G).\]
	Then
	\[\overline{\sigma}( zZ_i(G)) \in Z_{i+1}(G)/Z_i(G). \]
	Namely,
	\[\sigma(z)Z_i(G) \in  Z_{i+1}(G)/Z_i(G).\]
	So 
  	\[\sigma(z) \in Z_{i+1}(G).\]
	Thus,
  	\[\sigma\left(Z_{i+1}(G)\right) \subset Z_{i+1}(G).\]
	By the Theorem 4.4.8, $Z_{i+1}(G)$ is a characteristic subgroup of $G$.\\
	So the assumption also holds for $Z_{i+1}(G)$.\\
	Thus, we conclude that $Z_i(G)$ is a characteristic subgroup of $G$ for each $i$.
 \end{proof}
\end{exer}

\begin{exer}[6.1.6]
Let $G$ be a group. 
Prove that $G/Z(G)$ is nilpotent if and only if $G$ is nilpotent.
\end{exer}
\begin{proof}
  We first find the relationship between $({G/Z(G)})^n$ and $G^n$.\\
    $(G/Z(G))^0 = (G/Z(G)$.\\
	 Since $Z(G) \unlhd G$,
	\begin{align*}
  	  ({G/Z(G)})^1 &= [G/Z(G),(G/Z(G))^0] \\ 
  	  			 &= [G/Z(G),G/Z(G)] \\
  	  		 	 &= \left\langle [hZ(G), kZ(G)]|hZ(G),kZ(G) \in G/Z(G)\right\rangle \\
  	  			 &= \langle (hZ(G))^{-1}(kZ(G))^{-1}(hZ(G))(kZ(G)) | hZ(G),kZ(G) \in G/Z(G) \rangle\\
  	  			 &=\langle (h^{-1}Z(G))(k^{-1}Z(G))(hZ(G))(kZ(G)) |  hZ(G),kZ(G) \in G/Z(G) \rangle \\
  	  			 &=\langle h^{-1}k^{-1}hkZ(G) | h,k \in G \rangle.\\
  	  			 &=[G,G]/Z(G)\\
  	  			 &=G^1 / Z(G) 
	\end{align*}
	Then 
	\begin{align*}
	({G/Z(G)})^2 &=[G/Z(G),({G/Z(G)})^1] \\
	  			 &=[G/Z(G), G^1 Z(G)]\\
				 &= \left\langle [hZ(G), kZ(G)]|hZ(G) \in G/Z(G), kZ(G) \in G^1/ Z(G)\right\rangle \\
  	  			 &=\langle h^{-1}k^{-1}hkZ(G) | h \in G, k \in G^1 \rangle.\\
  	  			 &=[G,G^1]/Z(G).\\
  	  			 &=G^2/Z(G).
	\end{align*}
	We guess $({G/Z(G)})^n = G^n/Z(G)$.\\
	We will show it by induction.\\
	We have shown the basic steps.\\
	\textbf{Induction steps: }\\
	Assume $({G/Z(G)})^n = G^n/Z(G)$.
	\begin{align*}
	  ({G/Z(G)})^{n+1} &=[G/Z(G),({G/Z(G)})^n] \\
	  			 &=[G/Z(G), G^n/Z(G)]\\
				 &= \left\langle [hZ(G), kZ(G)]|hZ(G) \in G/Z(G), kZ(G) \in G^n /Z(G)\right\rangle \\
  	  			 &=\langle h^{-1}k^{-1}hkZ(G) | h \in G, k \in G^n \rangle.\\
  	  			 &=[G,G^n]/Z(G).\\
  	  			 &=G^{n+1}/Z(G).
	\end{align*}
	So the assumption holds for the $n+1$ case.\\
	Thus, for $n \in \bbn$,
	\[({G/Z(G)})^n = G^n/Z(G).\]
	"$\Leftarrow$". Assume $G$ is nilpotent.\\
	Then $G^m = e_G$ for some $m \geq 0$.\\
	So
	\[({G/Z(G)})^m = G^m/Z(G) = Z(G).\]
	So $G/Z(G)$ is nilpotent.\\
	"$\Rightarrow$". Assume $G/Z(G)$ is nilpotent.\\
	Then $(G/Z(G))^n = Z_G$ for some $n \geq 0$.\\
	So
	\[G^n/Z(G) = ({G/Z(G)})^n = Z(G).\]
	As a result, we have 
	\[G^n \subset Z(G).\]
	Therefore,
	\begin{align*}
	 	G^{n+1} &= [G,G^n] \\
				&=\left\langle [h, k]|h \in G, k \in G^n \right\rangle \\
				&=\left\langle h^{-1}k^{-1}hk|h \in G, k \in G^n \right\rangle \\
				&=\left\langle h^{-1}k^{-1}kh|h \in G, k \in G^n \right\rangle \\
				&=\left\langle e_G|h \in G, k \in G^n \right\rangle \\
	  			&=\langle e_G \rangle \\
	  			&= e_G,
  	\end{align*}
  	since $G^n \subset Z(G)$.\\
	Thus, $G$ is nilpotent.\\
Therefore, $G/Z(G)$ is nilpotent if and only if $G$ is nilpotent.
 \end{proof}

\end{document}




