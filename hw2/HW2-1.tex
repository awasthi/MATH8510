\documentclass{amsart}

% PACKAGES

\usepackage{amsmath}
\usepackage{amsfonts}
\usepackage{amssymb,enumerate}
\usepackage{amsthm,stmaryrd}
\usepackage[all]{xy}
\usepackage{hyperref}

%\theoremstyle{definition}
%\newtheorem{exer}{Exercise}

\newcommand{\bbr}{\mathbb{R}}
\newcommand{\bbc}{\mathbb{C}}
\newcommand{\bbz}{\mathbb{Z}}
\newcommand{\bbq}{\mathbb{Q}}
\newcommand{\bbn}{\mathbb{N}}
\newcommand{\be}{\mathbf{e}}
\newcommand{\ba}{\mathbf{a}}
\newcommand{\fm}{\mathfrak{m}}
\newcommand{\Hom}{\operatorname{Hom}}
\renewcommand{\ker}{\operatorname{Ker}}
\newcommand{\im}{\operatorname{Im}}
\newcommand{\xra}{\xrightarrow}
\newcommand{\wti}{\widetilde}

\theoremstyle{plain}
\newtheorem{lem}{Lemma}
\newtheorem{cor}[lem]{Corollary}
\newtheorem{prop}[lem]{Proposition}
\newtheorem{thm}[lem]{Theorem}
\newtheorem{conj}[lem]{Conjecture}
\newtheorem{intthm}{Theorem}
\renewcommand{\theintthm}{\Alph{intthm}}

\theoremstyle{definition}
\newtheorem{defn}[lem]{Definition}
\newtheorem{ex}[lem]{Example}
\newtheorem{question}[lem]{Question}
\newtheorem{questions}[lem]{Questions}
\newtheorem{problem}[lem]{Problem}
\newtheorem{disc}[lem]{Remark}
\newtheorem{rmk}[lem]{Remark}
\newtheorem{construction}[lem]{Construction}
\newtheorem{notn}[lem]{Notation}
\newtheorem{fact}[lem]{Fact}
\newtheorem{para}[lem]{}
\newtheorem{exer}[lem]{Exercise}
\newtheorem{remarkdefinition}[lem]{Remark/Definition}
\newtheorem{notation}[lem]{Notation}
\newtheorem{step}{Step}
\newtheorem{convention}[lem]{Convention}
\newtheorem*{Convention}{Convention}
\newtheorem{assumption}[lem]{Assumption}

\newcommand{\fmn}{F^{m\times n}}
\newcommand{\fnn}{F^{n\times n}}
\newcommand{\col}{\operatorname{Col}}
\newcommand{\row}{\operatorname{Row}}
\newcommand{\Span}{\operatorname{Span}}	
\newcommand{\rank}{\operatorname{rank}}	
\begin{document}

\noindent MATH 8510, Abstract Algebra I \\
Fall 2016\\
Exercises 2-1\\
Shuai Wei

\

%\noindent
%Throughout this homework set, let $F$ be a field
%
%
%\

\begin{exer}
Let $n\in\bbn$, and
consider the complex number 
$$e^{2\pi i/n}=\cos(2\pi/n)+\sin(2\pi/n)i\neq 0$$
as an element of the 
multiplicative abelian group $\bbc^\times$.
Compute the order $|e^{2\pi i/n}|$. \\
$$e=e^{2k\pi i/n} = 1_{\bbr} + 0_{\bbr}i \in \bbc^\times.$$
where $k \in {\bbz}$.
$$\forall n \in \bbn, \; e^{2\pi i/n} e = e^{2\pi i/n}(1_{\bbr} + 0_{\bbr}i) = e^{2\pi i/n}1_{\bbr} = e^{2\pi i/n}.$$
\begin{enumerate}
\item
	If $n=1, e^{2\pi i/n} = e^{2\pi i} = 1_{\bbr} + 0_{\bbr}i = e$, so $|e^{2\pi i/n}|= n=1$. \\
	So the order $ |e^{2\pi i/n}|$ is 1.
\item
	If $n > 1$, \\
		$$(e^{2\pi i/n})^1 = \cos(2\pi/n)+\sin(2\pi/n)i \neq 1_{\bbr} + 0_{\bbr}i;$$
		For $1<m<n$,
		$$(e^{2\pi i/n})^m = \cos(2m\pi/n)+\sin(2m\pi/n)i \neq 1_{\bbr} + 0_{\bbr}i;$$
		since $m/n \not\in \bbz$. \\
		But 
		$$(e^{2\pi i/n})^n = e^{2\pi i} =  \cos(2\pi)+\sin(2\pi)i = 1_{\bbr} + 0_{\bbr}i = e.$$
		So the order $ |e^{2\pi i/n}|$ is $n$.
\end{enumerate}
In summary, the order $ |e^{2\pi i/n}|$ is $n$. 

\end{exer}

\begin{exer}
Let $A$ and $B$ be groups. Prove that $A$ and $B$ are both abelian if and only if the cartesian product
$A\times B$ is abelian.
\begin{proof}
	$\forall a1,a2 \in A, b1,b2 \in B$, we have $(a1,b1), (a2,b2) \in A \times B$.\\ Then $(a1,b1), (a2,b2)$ are abitrary two elements from $A \times B.$ \\
	By definition, $(a1,b1)(a2,b2) = (a1a2,b1b2), (a2,b2)(a1,b1)=(a2a1,b2b1)$.
	$$A\times B \text{ is abelian group.}$$ 
	$$\Leftrightarrow (a1,b1)(a2,b2) = (a2,b2)(a1,b1).$$
	$$\Leftrightarrow (a1a2,b1b2) = (a2a1,b2b1).$$
	$$\Leftrightarrow a1a2 = a2a1 \text{ and } b1b2 = b2b1.$$
	$$\Leftrightarrow\text{A and B are both abelian.}$$	
\end{proof}
\end{exer}

\begin{exer}
Let $G$ be a group, and let $x\in G$ be an element with finite order $n$.
Prove that the elements $1,x,x^2,\ldots,x^{n-1}$ are distinct in $G$.
Deduce that $|x|\leq|G|$.
\begin{proof}
	 $x \neq 0$ since $0^n = 0 \neq 1$ for positive integer $n$, namely, 0 has not finite order.\\
	 Then we can write $1$ as $x^0$. \\
	 Assume there exists $0\leq i<j<n$ such that $x^i = x^j$.\\
	 Let $(x^i)^{-1} = x^{-i}$ be the inverse of $x^i$.\\
	 Then multiply $x^{-i}$ in two sides of $x^i = x^j$, we have $1=x^{i-i} = x^{j-i}$. \\
	 Thus $j-i >=n$ since $n$ is the order of $x$. \\
	 It is a contradiction since $j-i \leq n-1$ by the assumption $0\leq i<j<n$.\\
	 Therefore, the elements $1,x,x^2,\ldots,x^{n-1}$ are distinct in $G$.\\
	 As a result, we get $G$ has at least $n$ distinct elements, $|G| \geq n = |x|$, which completes the proof.
\end{proof}

\end{exer}


\end{document}


\begin{exer}
\begin{enumerate}[(a)]
\item 
\end{enumerate}
\end{exer}















