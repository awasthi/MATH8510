\documentclass{article}

% PACKAGES

\usepackage{amsmath}
\usepackage{amsfonts}
\usepackage{amssymb,enumerate}
\usepackage{amsthm,stmaryrd}
\usepackage[all]{xy}
\usepackage{hyperref}

%\theoremstyle{definition}
%\newtheorem{exer}{Exercise}

\newcommand{\bbr}{\mathbb{R}}
\newcommand{\bbc}{\mathbb{C}}
\newcommand{\bbz}{\mathbb{Z}}
\newcommand{\bbq}{\mathbb{Q}}
\newcommand{\bbn}{\mathbb{N}}
\newcommand{\be}{\mathbf{e}}
\newcommand{\ba}{\mathbf{a}}
\newcommand{\fm}{\mathfrak{m}}
\newcommand{\Hom}{\operatorname{Hom}}
\renewcommand{\ker}{\operatorname{Ker}}
\newcommand{\im}{\operatorname{Im}}
\newcommand{\xra}{\xrightarrow}
\newcommand{\wti}{\widetilde}

\theoremstyle{plain}
\newtheorem{lem}{Lemma}
\newtheorem{cor}[lem]{Corollary}
\newtheorem{prop}[lem]{Proposition}
\newtheorem{thm}[lem]{Theorem}
\newtheorem{conj}[lem]{Conjecture}
\newtheorem{intthm}{Theorem}
\renewcommand{\theintthm}{\Alph{intthm}}

\theoremstyle{definition}
\newtheorem{defn}[lem]{Definition}
\newtheorem{ex}[lem]{Example}
\newtheorem{question}[lem]{Question}
\newtheorem{questions}[lem]{Questions}
\newtheorem{problem}[lem]{Problem}
\newtheorem{disc}[lem]{Remark}
\newtheorem{rmk}[lem]{Remark}
\newtheorem{construction}[lem]{Construction}
\newtheorem{notn}[lem]{Notation}
\newtheorem{fact}[lem]{Fact}
\newtheorem{para}[lem]{}
\newtheorem{exer}[lem]{Exercise}
\newtheorem{remarkdefinition}[lem]{Remark/Definition}
\newtheorem{notation}[lem]{Notation}
\newtheorem{step}{Step}
\newtheorem{convention}[lem]{Convention}
\newtheorem*{Convention}{Convention}
\newtheorem{assumption}[lem]{Assumption}

\newcommand{\fmn}{F^{m\times n}}
\newcommand{\fnn}{F^{n\times n}}
\newcommand{\col}{\operatorname{Col}}
\newcommand{\row}{\operatorname{Row}}
\newcommand{\Span}{\operatorname{Span}}	
\newcommand{\rank}{\operatorname{rank}}	
\newcommand{\OO}[1]{\mathcal{O}_{#1}}
\begin{document}

\noindent MATH 8510, Abstract Algebra I \\
Fall 2016\\
Exercises 13-1\\
Collbrators: Dazhou Zhu, Xiaoyuan Liu\\
Name: Shuai Wei



\begin{exer}
Let $R$ be a commutative ring with identity.
\begin{enumerate}[(a)]
\item 
Let $A,B\subseteq R$ and set $I=(A)R$ and $J=(B)R$. 
Prove that $I+J$ is generated by $A\cup B$.
\begin{proof}
    \begin{enumerate}[(1)]
    \item
        If $A = B=\emptyset$, we have $A\cup B = \emptyset$ and $I=J = \{0\}$.\\
        Then
        \[I+J = \{0\} = (\emptyset)R = (A\cup B)R. \]
        So $I+J$ is generated by $A\cup B$.
    \item 
        If $A= \emptyset$ and $B \neq \emptyset$, we have $A\cup B = B$ and 
            \[I+J = (\emptyset)R + (B)R = \{0\}+(B)R = (B)R =(A\cup B)R. \]
        So $I+J$ is generated by $A\cup B$.
    \item
        Assume $A \neq \emptyset$ and $b \neq \emptyset$.\\
        Since $(A)R = I \leq R$ and $(B)R = J \leq R$,\\
    we have 
    \[I + J \leq R.\]
    Since $A \subseteq A\cup B$, 
    \[I=(A)R \subseteq (A\cup B)R.\]
    Similarly, 
    \[J \subseteq (A\cup B)R.\]
    By the definition of $I+J$, we have
    \begin{equation}\label{eq1}
    I+J \subseteq (A\cup B)R.
    \end{equation}
    Since $R$ is CRW1,
    \[(A\cup B)R = \{\sum_{i}^{\text{finite}}c_ir_i\mid c_i \in A\cup B, r_i \in R\}.\]
    Let $x \in (A\cup B)R$, then $\exists \ N \in \bbn$ and $c_i \in A\cup B$ and $r_i \in R$ for $i = 1,2,\cdots,N$ such that 
    \begin{align*}
      x  &= \sum_{i}^{N} c_ir_i.
    \end{align*}
     Without loss of generality, assume $\exists \ c_i \in A$ for some integer $i$ between $1$ and $N$.\\
    Rearrange $\{c_ir_i,i=1,\cdots,N\}$ such that $c_i \in A$ for $i = 1,\cdots,M$ and $c_i \in B$ for $i = M+1,\cdots,N$, where $M \in \bbn$ and $1\leq M \leq N$.\\
Then
   \begin{align*}
       x &= \sum_{i}^{M} c_ir_i + \sum_{i=M+1}^{N} c_ir_i \\
       &\in (A)R + (B)R\\
       &=I+J.
   \end{align*}
So
\begin{equation}\label{eq2}
(A\cup B)R \subseteq I+J.
\end{equation}
Thus, by \eqref{eq1} and \eqref{eq2}, we have
\[I + J = (A\cup B)R.\]
Therefore, $I+J$ is generated by $A\cup B$.
\end{enumerate}
\end{proof}
\item 
Prove that if $I$ and $J$ are finitely generated ideals of $R$, then $I+J$ is also finitely generated.
\begin{proof}
    Assume the ideal $I$ of $R$ is finitely generated by the set $A = \{a_1,a_2,\cdots,a_m\}$, where $a_1,\cdots,a_m \in R$,\\
    and the ideal $J$ of $R$ is finitely generated by the set $B=\{b_1,b_2,\cdots,b_n\}$, where $b_1,\cdots,b_m \in R$.\\
    Then
    \[I= (a_1,\cdots,a_m)R = (A)R\]
    and 
    \[J = (b_1,\cdots,b_n)R = (B)R.\]
    By part (a), we have $I+J$ is generated by $A\cup B$.\\
    Since $A\cup B$ is a finite set, $I+J$ is finitely generated.
\end{proof}
\end{enumerate}
\end{exer}

\begin{exer}
Let $R$ be a non-zero commutative ring with identity, and let $z\in R$.
Assume that $z$ is not nilpotent.
Use the following steps to prove that there is a prime ideal of $R$ that does not contain $z$.
\begin{enumerate}[(a)]
\item Set $\Sigma:=\{I\leq R\mid 1,z,z^2,\ldots\notin I\}$, partially ordered by inclusion. Prove that $\Sigma\neq\emptyset$ and that every chain in $\Sigma$ has an upper bound in $\Sigma$.
Use Zorn's Lemma to conclude that $\Sigma$ has a maximal element $K$.
\begin{proof}
    Let $I = \{0\}$, then $I \leq R$.\\
    Since $z$ is not nilpotent, $z^n \neq 0$, $\forall \ n \in \bbz^{\geq 0}$.\\
    So $z^n \not\in I, \forall \ n \in \bbz^{\geq 0}$.\\
    Thus, $I \in \Sigma$ and then $\Sigma \neq \emptyset$.\\
    Next we show every chain in $\Sigma$ has an upper bound in $\Sigma$.\\
    Let $\mathcal{C}$ be a chain in $\Sigma$.\\
    Set \[I= \bigcup_{J \in \mathcal{C}}J.\]
    Since $(\Sigma, \subseteq)$ is a poset,
    \[I \leq R.\]
    Suppose there exists at least one $z^n \in I$ for some $n \in \bbz^{\geq 0}$.\\
    Then $z^n \in J$ for some $J \in \mathcal{C}\subseteq \Sigma$.\\
    Since $J \in \Sigma$, $z^n \not\in J, \forall \ n \in \bbz^{\geq 0}$.\\
    So there is a contradiction.\\
    Then
    \[z^n \not\in I, \forall \ n \in \bbz^{\geq 0}.\]
    So 
    \[I  \in \Sigma.\]
    Also 
    \[\forall \ J \in \mathcal{C}, J \subseteq I.\]
    Thus, $I$ is an upper bound for $\mathcal{C}$ in $\Sigma$.\\
    By Zorn's lemma, $\Sigma$ has a maximal element $K$.
\end{proof}
\item Prove that $K$ is prime as follows.
\begin{enumerate}[(1)]
\item 
Suppose that $r,s\in R-K$ are such that $rs\in K$. Show that $K\subsetneq K+rR\leq R$
and $K\subsetneq K+sR\leq R$.
\begin{proof}
    Since $0_R \in R$, 
    \[K = K+r0_R \subseteq K+rR.\]
    Assume $K = K + rR$.\\
    Since $1_R \in R$, 
    \[K+r=K+r1_R \subseteq K+rR = K.\]
    By part (a), we already have $K \leq R$, so $r \in K$.\\
    As a result, there is a contradiction since $r \in R-K$ by assumption.\\
    Therefore, 
    \begin{equation}\label{eq2.1}
        K \subsetneq K +rR.
    \end{equation}
    Since $R$ is CRW1, $rR = (r)R \leq R$.\\
   Also, $K \leq R$.\\
   So 
   \begin{equation}\label{eq2.2}
        K+rR \leq R.
    \end{equation}
    By \eqref{eq2.1} and \eqref{eq2.2}, 
   \[K \subsetneq K +rR \leq R.\]
   Similarly, 
   \[K \subsetneq K +sR \leq R.\]

\end{proof}
\item Conclude that there are $m,n\in\bbz^{\geq 0}$ such that $z^m\in K+rR$ and $z^n\in K+sR$.
\begin{proof}
    Assume $z^m \not\in K+rR, \forall \ m \in \bbz^{\geq 0}$.\\
    Since $K+rR \leq R$, we have 
    \[K+rR \in \Sigma.\]
    Since $K$ is the maximal element of $\Sigma$, $K+rR \subseteq K$.\\
    So there is a contradiction since $K \subsetneq K+rR$.\\
    Thus, $\exists \ m \in \bbz^{\geq 0}$ such that $z^m \in K+rR$.\\
    Similarly, $\exists \ n \in \bbz^{\geq 0}$ such that $z^n \in K+sR$.\\
\end{proof}
\item Deduce that $z^{m+n}\in K$,  derive a contradiction, and conclude that $K$ is prime.
    \begin{proof}
        Since $z^m \in K+rR$ and $z^n \in K+sR$, there exists $k_1,k_2 \in K$ and $p_1,p_2 \in R$ such that $z^m = k_1 + rp_1$ and $z^n = k_2+ sp_2$.\\
        Since $R$ is CRW1,
        \begin{align*}
            z^{m+n} &= (z^m) (z^n) \\
            &= (k_1 + rp_1)(k_2+sp_2) \\
            &=k_1k_2+k_1(sp_2) + k_2(rp_1) + rs(p_1p_2)  
        \end{align*}
        Since $r,s,p_1,p_2 \in R$, we have
        \[sp_2,rp_1,p_1p_2 \in R.\]
        Since $K \leq R$ and $k_1,k_2,rs \in K$, we have
        \[k_1k_2, k_1(sp_2), k_2(rp_1), rs(p_1p_2) \in K.\]
        Then
        \[k_1k_2+k_1(sp_2) + k_2(rp_1) + rs(p_1p_2) \in K.\]
        So for $m,n \in \bbz^{\geq 0}$, we have
        \[z^{m+n} \in K.\]
        Since $K \in \Sigma$, we have $z^n \not\in K, \forall \ n \in \bbz^{\geq 0}$, which is contradicted by $z^{m+n} \in K$.\\
        So our assumption does not holds.\\
        Thus, $\forall \ r,s\in R-K$, $rs \not\in K$.\\
        Henceforth, $K$ is prime.\\
        As a result, there is a prime ideal $K$ of $R$ that does not contain $z$.
    \end{proof}
\end{enumerate}
\end{enumerate}
\end{exer}

\begin{exer}
Let $i=\sqrt{-1}\in\bbc$, and consider the following subrings of $\bbc$.
\begin{align*}
\bbz[i]&:=\{a+bi\mid a,b\in\bbz\}
\\
\bbq[i]&:=\{a+bi\mid a,b\in\bbq\}
\end{align*}
Prove that $\bbq[i]$ is isomorphic to the field of fractions of $\bbz[i]$.
\begin{proof}
    Let $R = Z[i]$ and $S = Q[i]$.\\
    By the Theorem 7.5.9, there exists well-defined ring monomorphsim 
    \begin{align*}
        \rho: R &\to D^{-1}R \\
        r &\to \frac{rd}{d} ,\; d\in D
    \end{align*}
    We know $S$ is a field in Chapter 1, so $S$ is an integral domain.\\
    By the subring test, we have $R$ is a subring of $S$.\\
    Also $1_S = 1_\bbr \in R$, we have $R$ is an integral domain.\\
    Let $D = R\setminus 0$, then $D^{-1}S$ is the field of fractions of $R$.\\
   Define
   \begin{align*}
       \phi: R &\to S \\
             r &\to r
   \end{align*}
   Since $\phi$ is an identity map from integral domain $R$ to integral domain $S$, \\
   it is a ring homomorphism.\\
   Also $S$ is CRW1 since it is an integral domain.\\
   Since $S$ is a field, it is a division ring and then $S^{\times} = S\setminus 0$.\\
   Since $R \subseteq S$,
   \[\phi(D) = D = R\setminus 0 \subseteq S\setminus 0 = S^{\times}.\]    By the Universal Mapping Proerty, there exists a unque ring homomorphism 
   \[\Phi:D^{-1}R \to S\]
   such that $\Phi\circ \rho = \phi$ and $\Phi(1) = 1$.



\end{proof}

\end{exer}

\begin{exer}
Let $R$ be an integral domain and consider the ring homomorphism $\psi\colon\bbz\to R$ given by $\psi(n)=n\cdot 1_R$.
(You do not need to show that this is a well-defined ring homomorphism.)
\begin{enumerate}[(a)]
\item Prove that $\ker(\psi)=0$ or $\ker(\psi)=p\bbz$ for some prime number $p\in\bbz$.
\item Prove that if $p$ is a prime number such that $\ker(\psi)=p\bbz$, then $R$ contains a finite field as a subring.
\item Prove that if $R$ is a field and $\ker(\psi)=0$, then $R$ has a subring $Q\cong\bbq$.
\end{enumerate}
\end{exer}


\end{document}

