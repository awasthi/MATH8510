\documentclass{amsart}

% PACKAGES

\usepackage{amsmath}
\usepackage{amsfonts}
\usepackage{amssymb,enumerate}
\usepackage{amsthm,stmaryrd}
\usepackage[all]{xy}
\usepackage{hyperref}

%\theoremstyle{definition}
%\newtheorem{exer}{Exercise}

\newcommand{\bbr}{\mathbb{R}}
\newcommand{\bbc}{\mathbb{C}}
\newcommand{\bbz}{\mathbb{Z}}
\newcommand{\bbq}{\mathbb{Q}}
\newcommand{\bbn}{\mathbb{N}}
\newcommand{\be}{\mathbf{e}}
\newcommand{\ba}{\mathbf{a}}
\newcommand{\fm}{\mathfrak{m}}
\newcommand{\Hom}{\operatorname{Hom}}
\renewcommand{\ker}{\operatorname{Ker}}
\newcommand{\im}{\operatorname{Im}}
\newcommand{\xra}{\xrightarrow}
\newcommand{\wti}{\widetilde}

\theoremstyle{plain}
\newtheorem{lem}{Lemma}
\newtheorem{cor}[lem]{Corollary}
\newtheorem{prop}[lem]{Proposition}
\newtheorem{thm}[lem]{Theorem}
\newtheorem{conj}[lem]{Conjecture}
\newtheorem{intthm}{Theorem}
\renewcommand{\theintthm}{\Alph{intthm}}

\theoremstyle{definition}
\newtheorem{defn}[lem]{Definition}
\newtheorem{ex}[lem]{Example}
\newtheorem{question}[lem]{Question}
\newtheorem{questions}[lem]{Questions}
\newtheorem{problem}[lem]{Problem}
\newtheorem{disc}[lem]{Remark}
\newtheorem{rmk}[lem]{Remark}
\newtheorem{construction}[lem]{Construction}
\newtheorem{notn}[lem]{Notation}
\newtheorem{fact}[lem]{Fact}
\newtheorem{para}[lem]{}
\newtheorem{exer}[lem]{Exercise}
\newtheorem{remarkdefinition}[lem]{Remark/Definition}
\newtheorem{notation}[lem]{Notation}
\newtheorem{step}{Step}
\newtheorem{convention}[lem]{Convention}
\newtheorem*{Convention}{Convention}
\newtheorem{assumption}[lem]{Assumption}

\newcommand{\fmn}{F^{m\times n}}
\newcommand{\fnn}{F^{n\times n}}
\newcommand{\col}{\operatorname{Col}}
\newcommand{\row}{\operatorname{Row}}
\newcommand{\Span}{\operatorname{Span}}	
\newcommand{\rank}{\operatorname{rank}}	
\newcommand{\OO}[1]{\mathcal{O}_{#1}}
\begin{document}

\noindent MATH 8510, Abstract Algebra I \\
Fall 2016\\
Exercises 13-1\\
Due date Thu 01 Dec 4:00PM

\



\begin{exer}
Let $R$ be a commutative ring with identity.
\begin{enumerate}[(a)]
\item 
Let $A,B\subseteq R$ and set $I=(A)R$ and $J=(B)R$. 
Prove that $I+J$ is generated by $A\cup B$.
\item 
Prove that if $I$ and $J$ are finitely generated ideals of $R$, then $I+J$ is also finitely generated.
\end{enumerate}
\end{exer}

\begin{exer}
Let $R$ be a non-zero commutative ring with identity, and let $z\in R$.
Assume that $z$ is not nilpotent.
Use the following steps to prove that there is a prime ideal of $R$ that does not contain $z$.
\begin{enumerate}[(a)]
\item Set $\Sigma:=\{I\leq R\mid 1,z,z^2,\ldots\notin I\}$, partially ordered by inclusion. Prove that $\Sigma\neq\emptyset$ and that every chain in $\Sigma$ has an upper bound in $\Sigma$.
Use Zorn's Lemma to conclude that $\Sigma$ has a maximal element $K$.
\item Prove that $K$ is prime as follows.
\begin{enumerate}[(1)]
\item 
Suppose that $r,s\in R-K$ are such that $rs\in K$. Show that $K\subsetneq K+rR\leq R$
and $K\subsetneq K+sR\leq R$.
\item Conclude that there are $m,n\in\bbz^{\geq 0}$ such that $z^m\in K+rR$ and $z^n\in K+sR$.
\item Deduce that $z^{m+n}\in K$,  derive a contradiction, and conclude that $K$ is prime.
\end{enumerate}
\end{enumerate}
\end{exer}






\end{document}


\begin{exer}
\begin{enumerate}[(a)]
\item 
\end{enumerate}
\end{exer}















