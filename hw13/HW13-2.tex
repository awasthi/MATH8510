\documentclass{amsart}

% PACKAGES

\usepackage{amsmath}
\usepackage{amsfonts}
\usepackage{amssymb,enumerate}
\usepackage{amsthm,stmaryrd}
\usepackage[all]{xy}
\usepackage{hyperref}

%\theoremstyle{definition}
%\newtheorem{exer}{Exercise}

\newcommand{\bbr}{\mathbb{R}}
\newcommand{\bbc}{\mathbb{C}}
\newcommand{\bbz}{\mathbb{Z}}
\newcommand{\bbq}{\mathbb{Q}}
\newcommand{\bbn}{\mathbb{N}}
\newcommand{\be}{\mathbf{e}}
\newcommand{\ba}{\mathbf{a}}
\newcommand{\fm}{\mathfrak{m}}
\newcommand{\Hom}{\operatorname{Hom}}
\renewcommand{\ker}{\operatorname{Ker}}
\newcommand{\im}{\operatorname{Im}}
\newcommand{\xra}{\xrightarrow}
\newcommand{\wti}{\widetilde}

\theoremstyle{plain}
\newtheorem{lem}{Lemma}
\newtheorem{cor}[lem]{Corollary}
\newtheorem{prop}[lem]{Proposition}
\newtheorem{thm}[lem]{Theorem}
\newtheorem{conj}[lem]{Conjecture}
\newtheorem{intthm}{Theorem}
\renewcommand{\theintthm}{\Alph{intthm}}

\theoremstyle{definition}
\newtheorem{defn}[lem]{Definition}
\newtheorem{ex}[lem]{Example}
\newtheorem{question}[lem]{Question}
\newtheorem{questions}[lem]{Questions}
\newtheorem{problem}[lem]{Problem}
\newtheorem{disc}[lem]{Remark}
\newtheorem{rmk}[lem]{Remark}
\newtheorem{construction}[lem]{Construction}
\newtheorem{notn}[lem]{Notation}
\newtheorem{fact}[lem]{Fact}
\newtheorem{para}[lem]{}
\newtheorem{exer}[lem]{Exercise}
\newtheorem{remarkdefinition}[lem]{Remark/Definition}
\newtheorem{notation}[lem]{Notation}
\newtheorem{step}{Step}
\newtheorem{convention}[lem]{Convention}
\newtheorem*{Convention}{Convention}
\newtheorem{assumption}[lem]{Assumption}

\newcommand{\fmn}{F^{m\times n}}
\newcommand{\fnn}{F^{n\times n}}
\newcommand{\col}{\operatorname{Col}}
\newcommand{\row}{\operatorname{Row}}
\newcommand{\Span}{\operatorname{Span}}	
\newcommand{\rank}{\operatorname{rank}}	
\newcommand{\OO}[1]{\mathcal{O}_{#1}}
\begin{document}

\noindent MATH 8510, Abstract Algebra I \\
Fall 2016\\
Exercises 13-2\\
Due date Thu 01 Dec 4:00PM

\



\begin{exer}
Let $i=\sqrt{-1}\in\bbc$, and consider the following subrings of $\bbc$.
\begin{align*}
\bbz[i]&:=\{a+bi\mid a,b\in\bbz\}
\\
\bbq[i]&:=\{a+bi\mid a,b\in\bbq\}
\end{align*}
Prove that $\bbq[i]$ is isomorphic to the field of fractions of $\bbz[i]$.
\end{exer}

\begin{exer}
Let $R$ be an integral domain and consider the ring homomorphism $\psi\colon\bbz\to R$ given by $\psi(n)=n\cdot 1_R$.
(You do not need to show that this is a well-defined ring homomorphism.)
\begin{enumerate}[(a)]
\item Prove that $\ker(\psi)=0$ or $\ker(\psi)=p\bbz$ for some prime number $p\in\bbz$.
\item Prove that if $p$ is a prime number such that $\ker(\psi)=p\bbz$, then $R$ contains a finite field as a subring.
\item Prove that if $R$ is a field and $\ker(\psi)=0$, then $R$ has a subring $Q\cong\bbq$.
\end{enumerate}
\end{exer}






\end{document}


\begin{exer}
\begin{enumerate}[(a)]
\item 
\end{enumerate}
\end{exer}















