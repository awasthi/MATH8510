\documentclass{amsart}

% PACKAGES

\usepackage{amsmath}
\usepackage{amsfonts}
\usepackage{amssymb,enumerate}
\usepackage{amsthm,stmaryrd}
\usepackage[all]{xy}
\usepackage{hyperref}

%\theoremstyle{definition}
%\newtheorem{exer}{Exercise}

\newcommand{\bbr}{\mathbb{R}}
\newcommand{\bbc}{\mathbb{C}}
\newcommand{\bbz}{\mathbb{Z}}
\newcommand{\bbq}{\mathbb{Q}}
\newcommand{\bbn}{\mathbb{N}}
\newcommand{\be}{\mathbf{e}}
\newcommand{\ba}{\mathbf{a}}
\newcommand{\fm}{\mathfrak{m}}
\newcommand{\Hom}{\operatorname{Hom}}
\renewcommand{\ker}{\operatorname{Ker}}
\newcommand{\im}{\operatorname{Im}}
\newcommand{\xra}{\xrightarrow}
\newcommand{\wti}{\widetilde}

\theoremstyle{plain}
\newtheorem{lem}{Lemma}
\newtheorem{cor}[lem]{Corollary}
\newtheorem{prop}[lem]{Proposition}
\newtheorem{thm}[lem]{Theorem}
\newtheorem{conj}[lem]{Conjecture}
\newtheorem{intthm}{Theorem}
\renewcommand{\theintthm}{\Alph{intthm}}

\theoremstyle{definition}
\newtheorem{defn}[lem]{Definition}
\newtheorem{ex}[lem]{Example}
\newtheorem{question}[lem]{Question}
\newtheorem{questions}[lem]{Questions}
\newtheorem{problem}[lem]{Problem}
\newtheorem{disc}[lem]{Remark}
\newtheorem{rmk}[lem]{Remark}
\newtheorem{construction}[lem]{Construction}
\newtheorem{notn}[lem]{Notation}
\newtheorem{fact}[lem]{Fact}
\newtheorem{para}[lem]{}
\newtheorem{exer}[lem]{Exercise}
\newtheorem{remarkdefinition}[lem]{Remark/Definition}
\newtheorem{notation}[lem]{Notation}
\newtheorem{step}{Step}
\newtheorem{convention}[lem]{Convention}
\newtheorem*{Convention}{Convention}
\newtheorem{assumption}[lem]{Assumption}

\newcommand{\fmn}{F^{m\times n}}
\newcommand{\fnn}{F^{n\times n}}
\newcommand{\col}{\operatorname{Col}}
\newcommand{\row}{\operatorname{Row}}
\newcommand{\Span}{\operatorname{Span}}	
\newcommand{\rank}{\operatorname{rank}}	
\newcommand{\OO}[1]{\mathcal{O}_{#1}}
\begin{document}

\noindent MATH 8510, Abstract Algebra I \\
Fall 2016\\
Exercises 8-1\\
Name: Shuai Wei\\
Collaborator: Xiaoyuan Liu, Daozhou Zhu

\

%\noindent
%Throughout this homework set, let $F$ be a field
%
%
%\

\begin{exer}[UMP: Universal Mapping Property]
Let $(G,+)$ be an abelian group and let $g_1,\ldots,g_t\in G$. 
For $i=1,\ldots,t$ let $e_i=(0,\ldots,0,1,0,\ldots,0)\in\bbz^t$ be the ``$i$th standard basis vector''.
\begin{enumerate}[(a)]
\item 
Prove that there exists a unique abelian group homomorphism $\phi\colon\bbz^t\to G$ such that $\phi(e_i)=g_i$ for $i=1,\ldots,t$.
\begin{proof}
  Let $z_i \in \bbz$ for $i = 1,\ldots,t$.\\
  Define $\phi$ as
  \begin{align*}
  	\phi: \bbz^t &\to G \\
  			(z_1,\dots,z_t) & \mapsto \sum_{i=1}^tz_ig_i 
  \end{align*}
  For $i = 1,\ldots,t$, $z_i\in \bbz$ and $g_i \in (G,+)$, so $z_ig_i \in G$.\\
  Then 
  \[\sum_{i=1}^tz_ig_i \in G.\]
  So $\phi$ is well-defined.\\
  At first, we verify that for $i = 1,\ldots,t$,
  \[\phi(e_i) = 0g_1 + ...0g_{i-1} + 1g_i + 0g_{i+1} + \ldots + 0g_{t} = g_i.\]
  We then show $\phi$ is a homomorphism.\\
  Let $x=(x_1,x_2,\ldots,x_t), y=(y_1,y_2,\ldots,y_t) \in \bbz^t$, where $x_i,y_i \in \bbz$ for i = $1,2,\ldots,t$.\\
  Since $G$ is abelian,
  \begin{align*}
  	\phi(x+y) &= \phi((x_1+y_1,x_2+y_2,\ldots,x_t+y_t)) \\
  			 &= \sum_{i=1}^t(x_i+y_i)g_i \\
  			 &= \sum_{i=1}^tx_ig_i+\sum_{i=1}^ty_ig_i \\
  			 &= \phi(x)+\phi(y).
  \end{align*}
  So $\phi$ is a homomorphism.\\ 
  Suppose there exists another abelian group homomorphism $\varphi\colon\bbz^t\to G$ such that 
  \[\varphi(e_i)=g_i, \text{ for } {i=1,\ldots,t}.\]
  Let $z = (z_1,z_2,\ldots,z_t) \in \bbz^t$, then $z_i \in \bbz$ for $i=1,2,\ldots,t$.\\
  Then
  \[z = \sum_{i=1}^{t}z_ie_i.\]
  Since $\phi$ and $\varphi$ are homomorphisms,
  \begin{align*} 
  	\phi (z) &=\sum_{i=1}^tz_ig_i \\
  			 &=\sum_{i=1}^tz_i\varphi(e_i) \\
  			 &=\sum_{i=1}^t\varphi(z_ie_i) \\
			 &= \varphi(z)
  \end{align*}
  Since $z \in \bbz ^t$ is arbitrary, $\phi = \varphi$.\\
  Thus, such abelian group homomorphism is unique.

\end{proof}
\item 
Prove that $\im(\phi)=\langle g_1,\ldots,g_t\rangle$.
In particular, $\phi$ is surjective if and only if $G=\langle g_1,\ldots,g_t\rangle$.
\begin{proof}
  Let $z = (z_1,z_2,\ldots,z_t) \in \bbz^t$, then $z_i \in \bbz$ for $i=1,2,\ldots,t$. \\
  Then
    \[z = \sum_{i=1}^{t}z_ie_i.\]
  By the definition of $\phi$,
  	\[ \phi (z) = \sum_{i=1}^tz_ig_i \in \langle g_1,g_2,\ldots,g_t\rangle, \]    
  	so 
  	\[ \im(\phi) \subset \langle g_1,g_2,\ldots,g_t\rangle.\]
  	On the other hand, let $x \in \langle g_1,g_2,\ldots,g_t\rangle$, then $\exists\ x_1, x_2,\ldots,x_t \in \bbz$ such that 
  	\[x = \sum_{i=1}^tg_i{x_i}.\]
	Let $y= (x_1,x_2,...,x_t)$, then 
	\[x=\phi (y) = \sum_{i=1}^{t}x_i g_i \in \im(\phi). \]
	So
	\[\langle g_1,g_2,\ldots,g_t\rangle \subset \im(\phi).\]
	Thus,
	\[ \im(\phi) = \langle g_1,g_2,\ldots,g_t\rangle.\]
	Then we show the second statement.\\
	"$\Leftarrow$". Assume $G= \langle g_1,g_2,\ldots,g_t\rangle $.\\
	Since $\im(\phi) = G$, $\phi$ is surjective.\\
	"$\Rightarrow$". Assume $\phi$ is surjective.\\
	We show it by contradiction.\\
	Suppose $\exists \ g \in G$ but $g \not\in \langle g_1,g_2,\ldots,g_t\rangle$. \\
	By the assumption that $\phi$ is surjective, \\
	so $\exists \ f \in \bbz^t$ such that $\phi(f) = g$, where $f=(f_1,f_2,\ldots,f_t)$ and $f_i \in \bbz$ for $i=1,2,\ldots,t$.\\
	Then we have
	\[g = \phi (f) = \sum_{i=1}^tg_i{f_i} \in \langle g_1,g_2,\ldots,g_t\rangle,\]
	which is contradicted by the other assumption that $g \not\in \langle g_1,g_2,\ldots,g_t\rangle$.\\
	Thus, if $g \in G$, then $g \in \langle g_1,g_2,\ldots,g_t\rangle$.\\
	Namely, $G \subset \langle g_1,g_2,\ldots,g_t\rangle = \im(\phi)$.\\
	Besides, $\im(\phi) \subset G$ by the definition of $\phi$.\\
	Therefore,
	\[G = \im(\phi) = \langle g_1,g_2,\ldots,g_t\rangle.\]
	In summary, $\phi$ is surjective if and only if $G=\langle g_1,\ldots,g_t\rangle$.
\end{proof}


\item 
Prove that the following conditions are equivalent.
\begin{enumerate}[(i)]
\item $G$ is finitely generated.
\item 
There is an integer $t\geq 0$ and an epimorphism $\phi\colon\bbz^t\to G$.
\item There is an integer $t\geq 0$ and a subgroup $K\leq\bbz^t$ such that $G\cong\bbz^t/K$.
\begin{proof}
  "(i)$\Rightarrow$ (ii)".\\
  Let $G = \langle g_1,g_2,\ldots,g_n\rangle$.\\
  Define 
  \begin{align*}
	\phi: \bbz^t &\to G \\
  	(z_1,\ldots,z_t) &\mapsto \sum_{i=1}^tz_ig_i 
  \end{align*}
  Then by part (a), we have $\phi$ is a homomorphism.\\
  Besides, since $G$ is finitely generated, by part (b), $\phi$ is surjective.\\
  So $\phi$ is an epimorphism.\\
  Thus, there is an integer $t= n$ and an epimorphism $\phi:\bbz^t \to G$.\\
  "(ii) $\Rightarrow$ (iii)".\\
  	Since $\phi$ is epimorphism, it is a homomorphism. \\
	So by the First Isomorphism Theorem, we have
	\[\im(\bbz_t) \cong \bbz_t/\ker{\bbz_t}.\]
	Since $\phi$ is epimorphism, it is surjective.\\
	So 
	\[\im(\bbz_t) = G.\]
	Then
	\[G \cong \bbz_t/\ker{\bbz_t}.\]
  	Thus, there exists $t \in \bbn$ and a subgroup $K = \ker{\bbz_t} \leq \bbz_t$ such that $G \cong \bbz_t/K$.\\
  	"(iii) $\Rightarrow$ (ii)".\\
  	Since the possilbe subgroups of $\bbz$ are \{0\} and $\bbz_n$ for $ n \in \bbn$ and $n \geq 1$,
  	it is obvious $K = \{e_{\bbz^t}\}$ and $(n_1\bbz)\times (n_2\bbz)\times \ldots \times (n_t\bbz)$ are the possible subgroup of $\bbz^t$, where $e_{\bbz^t}$ is a $t$-dimensional zero vector and $n_i \in \bbz, n_i\geq 1$ for $i=1,2,\ldots,t$.\\
  	We will show it by the following two cases.
  	\begin{enumerate}[(1)]
  		\item
		Let $K=\{e_{\bbz^t}\}$.\\ 
		Then
		\[G \cong \bbz^t/K \cong \bbz^t.\]
		Define
  		\begin{align*}
			\phi: \bbz^t &\to G \\
  		  (z_1,\ldots,z_t) &\mapsto \sum_{i=1}^tz_ig_i 
  		\end{align*}
  		Then $\phi$ is an isomorphism, and then it is an epimorphism.\\
  		By part (b), we have $G$ is finitely generated.\\
		\item
		  Let $K = (n_1\bbz)\times (n_2\bbz)\times \ldots \times (n_t\bbz)$.\\
		  Then $\bbz^t / K = (\bbz/n_1\bbz) \times (\bbz/n_2\bbz)\times \ldots \times (\bbz/n_t\bbz)$.\\
		  So $\left| \bbz^t/K \right| =  n_1n_2 \ldots n_t$.\\
		  Since 
		  \[G \cong \bbz^t/K\]
			\[|G| = n_1n_2\ldots n_t < \infty.\]
		  So $G$ is finitely generated.
	\end{enumerate}
	Thus, $G$ is finitely generated if there is an integer $t\geq 0$ and a subgroup $K\leq\bbz^t$ such that $G\cong\bbz^t/K$.
\end{proof}


\end{enumerate}
\item Let $s\leq t$ and let $n_1,\ldots,n_s\in\bbz$. 
Prove that 
there is an isomorphism $$\bbz^t/\langle n_1e_1,\ldots,n_se_s\rangle\cong(\bbz/n_1\bbz)\times\cdots\times(\bbz/n_s\bbz)\times\bbz^{t-s}.$$
\begin{proof}
  	Let $G = (\bbz/n_1\bbz)\times\cdots\times(\bbz/n_s\bbz)\times\bbz^{t-s}$.\\
  	Let $g_i = (\bar{0},\ldots,\bar{0},\bar{1},\bar{0},\ldots,\bar{0},0,\ldots,0)\in \bbz^t$ be the $i$th basis vetor with the $i$-th element $\bar{1}$ for $1 \leq i \leq s$.\\
  	Let $g_i = (\bar{0},\ldots,\bar{0},0,\ldots,0,1,0,\ldots,1)$ be the $i$th basis vector with the $i$-th element $1$ for $s+1 \leq i \leq t$.\\
  	Then it is obvious that $G = \langle g_1,\ldots,g_s,g_{s+1},\ldots,g_t \rangle$.\\
  Define $\phi$ as
  \begin{align*}
  	\phi: \bbz^t &\to G \\
  			(z_1,\dots,z_t) & \mapsto \sum_{i=1}^tz_ig_i 
  \end{align*}
	So by part (a), $\phi$ is a homomorphism.\\
	Since $G$ is finitely generated, by part (b),
	\[\im(\phi) = G.\]
	Next we show 
	\[\ker(\phi) = \langle n_1e_1,\ldots,n_se_s \rangle.\]
	$\forall h=(h_1,\ldots,h_t)\in \bbz_t$, where $h_i \in \bbz$ for $i = 1,2,\ldots,t$, we have
	\[\phi(h) = g_1{h_1}+\ldots +g_t{h_t}.\]
  	\begin{align*}
  	  h \in \ker(\phi) &\Leftrightarrow \phi(h) = e_G = (\bar{0},\ldots,\bar{0},0,\ldots,0) \\   
  	 			     				   	  &\Leftrightarrow g_1{h_1}+\ldots +g_t{h_t} = (\bar{0},\ldots,\bar{0},0,\ldots,0) \\
  	 			     				   	  &\Leftrightarrow (\bar{h_1},\ldots,\bar{h_s},h_{s+1},\ldots,h_t) = (\bar{0},\ldots,\bar{0},0,\ldots,0) \\
  	  					  			      &\Leftrightarrow h_1 \in (n_1\bbz),\ldots,h_s \in (n_s\bbz),h_{s+1} = 0,\ldots,h_{t} = 0 \\
  	  									  &\Leftrightarrow h  \in \langle n_1e_1,\ldots,n_se_s \rangle.
  	\end{align*}
    Thus,
    \[\ker(\phi) = \langle n_1e_1,\ldots,n_se_s \rangle.\]
    By the First Isomorphism Theorem, we have
    \[\bbz^t/\langle n_1e_1,\ldots,n_se_s\rangle\cong(\bbz/n_1\bbz)\times\cdots\times(\bbz/n_s\bbz)\times\bbz^{t-s}.\]
\end{proof}

\end{enumerate}
\end{exer}

\begin{exer}
\begin{enumerate}[(a)]
\item Let $a,b\in\bbz^+$ be relatively prime. Use Exercise~5-1\#3
to prove that $\bbz/(ab)\bbz\cong(\bbz/a\bbz)\times(\bbz/b\bbz)$.
\begin{proof}
 	 Since $(\bbz, +)$ is abelian and $a\bbz \leq \bbz$ and $b\bbz \leq \bbz$,
	 \[a\bbz \unlhd \bbz \text{ and } b\bbz \unlhd \bbz.\]
	 Since $a$ and $b$ are relatively prime, $(a,b) = 1$.\\
	 Then $\exists \ x,y\in \bbz$ such that
	 \[ax+by = 1.\]
	 Let $z \in \bbz$, then 
	 \[a(xz) + b(yz) = z\]
	 Since $x,y,z\in\bbz$, $xz,yz \in \bbz$.\\
	 Then 
	 $a(xz) \in a\bbz$,and $b(yz) \in b\bbz$.\\
	 Then 
	 \[z \in a\bbz + b\bbz.\]
	 So
	 \[\bbz \subset a\bbz + b\bbz.\]
	 Besides, 
	 \[ a\bbz + b\bbz \subset \bbz.\]
	 Thus,
	 \[\bbz = a\bbz + b\bbz.\]	
	 Since $a,b \in \bbz$, $\text{lcm}(a,b) = ab$.\\
	 So 
	 \[(a\bbz)\cap (b\bbz) = (ab)\bbz.\]
	 Therefore, according to the conclusion from the Exercise~5-1\#3, \\
	 we have
	 \[\bbz/(ab)\bbz\cong(\bbz/a\bbz)\times(\bbz/b\bbz).\]
\end{proof}

\item Let $m\in\bbz^+$, let $p_1,\ldots,p_m$ be distinct prime numbers, and let $e_1,\ldots,e_m\in\bbz^{\geq 0}$.
%and set
%$n=p_1^{e_1}\cdots p_m^{e_m}$. 
Prove that
$\bbz/n\bbz\cong\prod_{i=1}^m\bbz/p_i^{e_i}\bbz$.
\begin{proof}
We will show it by induction.
\begin{enumerate}[(1)]
	\item
		Basic step: when $m=1$, $n=p_1^{e_1}$, \\
		then it is obvious that 
		\[ \bbz / n\bbz \cong \bbz/p_1^{e_1}\bbz.\]
	\item
	  Inductive step: assume when $m=k$, we have $n = \prod_{i=1}^kp_i^{e_i}$ and 
	  \[ \bbz / n\bbz \cong \prod_{i=1}^k\bbz/p_i^{e_i}\bbz.\]
	  Let $p_{k+1}$ be a prime such that $p_1,...,p_k,p_{k+1}$ are distinct and $e_{k+1} \in \bbz^{\geq 0}$.\\
	  Then $\prod_{i=1}^kp_i^{e_i}$ and $p_{k+1}^{e_{k+1}}$ are relatively primes.\\
	  According to the conclusion from part (a), we have 
	  \[\bbz/ \left(\left( \prod_{i=1}^kp_i^{e_i} \right) p_{k+1}^{e_{k+1}} \right)\bbz\cong \left(\bbz/\left( \prod_{i=1}^kp_i^{e_i} \right) \bbz \right)\times \left(\bbz/ p_{k+1}^{e_{k+1}} \bbz\right).\]
	  Namely,
	\[\bbz/ \left(\left( \prod_{i=1}^{k+1}p_i^{e_i} \right) \right)\bbz\cong ( \bbz /n\bbz )\times \left(\bbz/ p_{k+1}^{e_{k+1}} \bbz\right).\]
	  By the inductive assumption, we know
	  \[ \bbz / n\bbz \cong \prod_{i=1}^k\bbz/p_i^{e_i}\bbz.\]
	  So
	  \[\bbz/ \left(\left( \prod_{i=1}^{k+1}p_i^{e_i} \right) \right)\bbz\cong \left( \prod_{i=1}^k\bbz/p_i^{e_i}\bbz \right)\times \left(\bbz/ p_{k+1}^{e_{k+1}} \bbz\right).\]
	  Namely,
	  \[\bbz/ \left(\left( \prod_{i=1}^{k+1}p_i^{e_i} \right) \right)\bbz\cong \left( \prod_{i=1}^{k+1}\bbz/p_i^{e_i}\bbz \right).\] 
	  Thus, the assumption also holds for $m=k+1$.\\ 
\end{enumerate}
    Therefore,
	\[ \bbz/n\bbz\cong\prod_{i=1}^m\bbz/p_i^{e_i}\bbz.\]
\end{proof}

\end{enumerate}
\end{exer}

\newpage


\begin{exer}[5.2]
Write a list of the non-isomorphic abelian groups of order 270 in terms of their elementary divisor decompositions. 
For each group in this list, write its invariant factor decomposition. \\
\textbf{Solution:}\\
Let $G$ be an abelian group of order 270.
\[270 = 2 \times 3^3\times 5.\] 
Since $3=3$, $3=1+2$ and $3=1+1+1$,
we have 3 non-isomorphic abelian groups of order 270 and they are
\begin{gather*}
(\bbz/27\bbz) \times (\bbz/2\bbz) \times (\bbz/5\bbz);\\
\left( \bbz/9\bbz \times \bbz/3\bbz \right)\times (\bbz/2\bbz) \times (\bbz/5\bbz);\\
\left( \bbz/3\bbz \times \bbz/3\bbz \times \bbz/3\bbz \right)\times (\bbz/2\bbz) \times (\bbz/5\bbz).
\end{gather*}
Next we compute their invariant factor decomposition.
\begin{enumerate}[(1)]
	\item
	  \begin{align*}
	 	(\bbz/27\bbz) \times (\bbz/2\bbz) \times (\bbz/5\bbz) &\cong \bbz/270\bbz      		
	  \end{align*}
	\item
	  \begin{align*}
	  	\left( \bbz/9\bbz \times \bbz/3\bbz \right)\times (\bbz/2\bbz) \times (\bbz/5\bbz) &\cong  \left( \bbz/9\bbz \times \bbz/2\bbz \times \bbz/5\bbz \right)\times (\bbz/3\bbz) \\
	  			&\cong \left( \bbz/90\bbz \right)\times (\bbz/3\bbz)
	  \end{align*}
	\item
	\begin{align*}
	  &\left( \bbz/3\bbz \times \bbz/3\bbz \times \bbz/3\bbz \right)\times (\bbz/2\bbz) \times (\bbz/5\bbz) \\
	  &\cong  \left( \bbz/3\bbz \times \bbz/2\bbz \times \bbz/5\bbz \right) \times (\bbz/3\bbz) \times (\bbz/3\bbz)  \\
	  &\cong \left( \bbz/30\bbz \right)\times (\bbz/3\bbz) \times (\bbz/3\bbz) 
	\end{align*}
\end{enumerate}
\end{exer}

\begin{exer}[5.4.11]
Let $H$ and $K$ be characteristic subgroups of a group $G$ such that $H\cap K=\{e\}$
and $G=HK$. Prove that $\operatorname{Aut}(G)\cong\operatorname{Aut}(H)\times\operatorname{Aut}(K)$.

\begin{proof}
	Since $H$ and $K$ be characteristic subgroups of a group $G$,\\
	\[H \unlhd G \text{ and } K \unlhd G.\]
	Besides, 
	\[H \cap K = \{e\}.\]
	Then by Theorem 5.4, we have
	\[H \times K \cong HK = G. \]
	Let $\sigma \in \operatorname{Aut}(H)$ and $\tau \in \operatorname{Aut}(K)$.\\
	Define $\sigma \times \tau $ as
	\begin{align*}
	  \sigma \times \tau: H\times K &\to H\times K \\ 
	  (h,k) &\mapsto (\sigma(h), \tau (k))  
	\end{align*}
	Then we show $\sigma \times \tau \in \operatorname{Aut}(H \times K)$.\\
	Let $(h_1,k_1),(h_2,k_2) \in H\times K$.\\
	Since $\sigma \in \operatorname{Aut}(H)$, and $\tau \in \operatorname{Aut}(K)$, $\sigma,\tau$ are homomorphisms.
	\begin{align*}
	  \sigma \times \tau \left((h_1,k_1)(h_2,k_2)\right) &= \sigma \times \tau(h_1h_2,k_1k_2) \\
	  													 &=\left(\sigma(h_1h_2),\tau(k_1k_2)\right) \\
	  													 &=\left(\sigma(h_1)\sigma(h_2),\tau(k_1)\tau(k_2)\right) \\
	  													 &=\left(\sigma(h_1),\tau(k_1)\right) \left(\sigma(h_2),\tau(k_2)\right)\\
	  													 &=\left(\sigma \times \tau(h_1,k_1)\right)\left(\sigma \times \tau(h_2,k_2)\right),
	\end{align*}
	Therefore, $\sigma \times \tau$ is a homomorphism.\\
	Let $(h,k) \in H \times K$ where $h \in H, k \in K$.\\
	Since $\sigma \in \operatorname{Aut}(H)$, and $\tau \in \operatorname{Aut}(K)$, $\sigma$ and $\tau$ are isomomorphsims. \\
	So $\sigma (h) = e_G \text{ and }\tau (k) = e_G$ if and only if $h = e_G$ and $k = e_G$.
	\begin{align*}
	  (h,k) \in \ker(\sigma \times \tau) & \Leftrightarrow \sigma \times \tau (h,k) =(e_G,e_G) \\
	  									 & \Leftrightarrow  (\sigma (h), \tau (k)) = (e_G,e_G) \\
	  									 & \Leftrightarrow  \sigma (h) = e_G \text{ and }\tau( k) = e_G \\
	  									 & \Leftrightarrow  h = e_G \text{ and } k = e_G \\
	  									 & \Leftrightarrow (h,k) = (e_G,e_G),
	\end{align*}
   so 
   \[\ker(\sigma \times \tau) = (e_G,e_G).\]
   So $\sigma \times \tau$ is 1-1.\\
   Let $(h^{\textprime},k^{\textprime}) \in (H,K)$, where $h \in H, k \in K$.\\
   We have shown before that $\sigma^{-1} \in \operatorname{Aut}(H)$ and $\tau^{-1} \in \operatorname{Aut}(K)$ when $\sigma \in \operatorname{Aut}(H)$ and $\tau \in \operatorname{Aut}(K)$.\\
   Then $\sigma^{-1}(h^{\textprime}) \in H$ and $\tau^{-1}(k^{\textprime}) \in K$.\\
   So 
   \[\left(\sigma^{-1}(h^{\textprime}),\tau^{-1}(k^{\textprime})\right) \in H \times K.\]
   Since by the definition of $\sigma \times \tau$, 
   \[\sigma \times \tau \left(\sigma^{-1}(h^{\textprime}), \tau^{-1}(k^{\textprime}) \right) = (h^{\textprime},k^{\textprime}),\]
   $\sigma \times \tau$ is onto.\\
   Therefore, 
   \[\sigma \times \tau \in \operatorname{Aut}(G).\] 
   Next we define $\phi$ as 
   \begin{align*}
	 \phi: \operatorname{Aut}(H) \times \operatorname{Aut}(K) &\to \operatorname{Aut}(H \times K) \\
	 					(\sigma, \tau) &\mapsto \sigma \times \tau
   \end{align*}
   Since we have show $\sigma \times \tau \in \operatorname{Aut}(H \times K)$, $\phi$ is well-defined.\\
   Let $(\sigma_1,\tau_1),(\sigma_2,\tau_2) \in \operatorname{Aut}(H) \times \operatorname{Aut}(K)$, where $\sigma_1,\sigma_2 \in \operatorname{Aut}(H)$ and $\tau_1,\tau_2 \in \operatorname{Aut}(K)$.\\
   Let $(h,k) \in H \times K$ where $h \in H, k \in K$.
   \begin{align*}
	 \left((\sigma_1\sigma_2) \times (\tau_1\tau_2)\right)(h,k) &= \left((\sigma_1\sigma_2)(h), (\tau_1\tau_2)(k) \right)\\	
	 							  								&=(\sigma_1(\sigma_2(h)), \tau_1(\tau_2(k)))\\
	 												  &=(\sigma_1 \times \tau_1)(\sigma_2(h),\tau_2(k))\\
	 												  &=(\sigma_1 \times \tau_1) \left((\sigma_2 \times \tau_2)(h,k)\right) \\
	 												  &=\left((\sigma_1 \times \tau_1) (\sigma_2 \times \tau_2)\right)(h,k), 
   \end{align*}
   so
   \[(\sigma_1\sigma_2)\times (\tau_1\tau_2) = (\sigma_1 \times \tau_1) (\sigma_2 \times \tau_2)\]
   Then
   \begin{align*}
   	 \phi\left((\sigma_1,\tau_1)(\sigma_2,\tau_2)\right) &= \phi (\sigma_1\sigma_2,\tau_1\tau_2)\\
   	 													 &=(\sigma_1\sigma_2) \times (\tau_1\tau_2) \\
   	 													 &=(\sigma_1 \times \tau_1) (\sigma_2 \times \tau_2) \\
   	 													 &=\left(\phi(\sigma_1,\tau_1)\right)\left(\phi(\sigma_2,\tau_2)\right).
   \end{align*}
   So $\phi$ is a homomorphism.\\
   Let $(\sigma,\tau) \in \operatorname{Aut}(H) \times \operatorname{Aut}(K)$, where $\sigma \in \operatorname{Aut}(H)$ and $\tau \in \operatorname{Aut}(K)$.\\
   Let $id_H$ and $id_K$ be the identity maps of $\operatorname{H}$ and $\operatorname{K}$, respectively.\\
   Let $id_{H\times K}$ be the identity map of $\operatorname{Aut}(H\times K)$.
   \begin{align*}
	 (\sigma,\tau) \in \ker(\phi) &\Leftrightarrow \phi(\sigma,\tau) = id_{H \times K} \\
	 							  &\Leftrightarrow \sigma\times \tau = id_{H \times K}\\
	 							  &\Leftrightarrow \sigma \times \tau = id_{H} \times id_{K} \\
	 							  &\Leftrightarrow (\sigma,\tau) = (id_{H},id_{K})
   \end{align*}
	So $\phi$ is 1-1.\\
	Let $\pi \in \operatorname{Aut}(H \times K)$.\\
	Define two maps $\pi_H:H\to H$ and $\pi_K: K\to K$ by $\left(\pi_H(h),1\right) = \pi(h,1)$ and $ \left(1,\pi_K(k)\right) = \pi(1,k)$.\\
	Repeat the similar processes as previous ones, \\
	we have $\pi_H$ and $\pi_K$ are well-defined and $\pi_H \in \operatorname{Aut}(H)$ and $\pi_K \in \operatorname{Aut}(K)$.\\
	Let $(h,k) \in H \times K$ where $h \in H, k \in K$.
	\begin{align*}
	  \pi(h,k) &= \pi((h,1)(1,k)) \\
	  		   &=\pi(h,1) \pi(1,k) \\
	  		   &=\left(\pi_H(h),1\right) \left(1,\pi_K(k)\right) \\
	  		   &=\left(\pi_H(h), \pi_K(k)\right) \\
	  		   &= \pi_H \times \pi_K (h,k),
	\end{align*}
	so $\pi = \pi_H \times \pi_K$.\\
	Thus, $\phi$ is onto.\\
	As a result, 
	\[\operatorname{Aut}(H) \times \operatorname{Aut}(K) \cong \operatorname{Aut}(H \times K) \]
   Since we have show 
   \[ H\times K \cong G,\] 
   \[\operatorname{Aut}(G) \cong \operatorname{Aut}(H \times K).\]
   Hence,
   \[\operatorname{Aut}(G) \cong \operatorname{Aut}(H) \times \operatorname{Aut}(K)\]
\end{proof}

Use this to prove that if $G$ is a finite abelian group, then $\operatorname{Aut}(G)$ is isomorphic to the direct product of the automorphism groups of its Sylow subgroups.
\begin{proof}
	Let $\{P_i\}_{i=1}^n$ be the collection of all the Sylow subgroups of $G_n$.\\
	Then $G_n =  P_1P_2\ldots P_n$.\\
	Since $G_n$ is abelian and $P_i \leq G_n$ for $1\leq i \leq n$,\\
	\[P_i \unlhd G_n.\]
	So the Sylow $|P_i|$-subgroup is unique for $1\leq i \leq n$. \\
	Thus, $P_i \cap P_j = \{e_{G_n}\}$ for $1 \leq i,j \leq n$ and $i\neq j$.\\
	Besides, by Corollary 4.5.6, $P_i$ is a characteristic subgroup of $G_n$ for $1\leq i \leq n$.\\ 
	We will show it by induction.\\
	\textbf{Basic steps:}\\
	When $n=1$, it is a trivial case since the only Sylow subgroup is $G_n$ and $ \operatorname{Aut}(G_n) \cong \operatorname{Aut}(G_n)$.\\
	We have just showed the case for $n=2$.\\
	\textbf{Inductive steps:}\\
	Assume 
	\[\operatorname{Aut}(G_n)  = \operatorname{Aut}(P_1P_2\ldots P_n) \cong \prod_{i=1}^{n} \operatorname{Aut}(P_i).\]
    Let $\{Q_i\}{i=1}^{n+1}$ be the collection of all the Sylow subgroups of $J_{n+1}$.\\
	Then $J_{n+1} =  Q_1Q_2\ldots Q_{n+1}$.\\
	Similarly, we have for $i=1,2,\ldots,n+1$,
	\[Q_i \unlhd J_{n+1}.\]
	For $1 \leq i,j \leq n+1$ and $i\neq j$,
	\[Q_i \cap Q_j = \{e_{J_{n+1}}\}.\]
	For $1\leq i \leq n+1$, $Q_i$ is a characteristic subgroup of $J_{n+1}$.\\
	Let $J_n = Q_1Q_2\ldots Q_n$.\\
	Then it is obvious that $\{Q_i\}_{i=1}^{n}$ is the collection of all the Sylow subgroups of $J_{n}$. \\
	Let $\sigma \in \operatorname{Aut}(J_{n+1})$.\\
	Then for $i = 1,2,\ldots,n$, $\sigma(Q_i)  = Q_i$ since $Q_i$ is a characteristic subgroup of $J_{n+1}$.\\ 
	Since $\sigma$ is a homomorphism,
	\begin{align*}
	 	\sigma(J_n) &= \sigma(Q_1Q_2\ldots Q_n)\\
	  				&= \sigma(Q_1)\sigma(Q_2)\ldots \sigma(Q_n) \\
	  				&=Q_1 Q_2\dots Q_n \\
	  			   &=J_n.
	\end{align*}
	So $J_n$ is a characteristic subgroup of $J_{n+1}$.\\
 	Since for $1 \leq i,j \leq n+1$ and $i\neq j$,
 		\[Q_i \cap Q_j = \{e_{J_{n+1}}\},\]
 	and $J_n =Q_1Q_2\ldots Q_n$, we have 
 	\[J_n \cap Q_{n+1} = \{e_{J_{n+1}}\}.\]
	We already have $Q_{n+1}$ is a characteristic subgroup of $J_{n+1}$.\\
	Besides,
	\[J_{n+1} = J_nQ_{n+1}.\]
	By the conclusion we just made,
	\[\operatorname{Aut}(J_{n+1}) \cong \operatorname{Aut}(J_n) \times \operatorname{Aut}(Q_{n+1}).\]
	By the inductive assumption, we have
	\[\operatorname{Aut}(J_n)  = \operatorname{Aut}(Q_1Q_2\ldots Q_n) \cong \prod_{i=1}^{n} \operatorname{Aut}(Q_i).\]
	Thus,
  	\[\operatorname{Aut}(J_{n+1}) \cong \left(\prod_{i=1}^{n} \operatorname{Aut}(Q_i) \right)\times \operatorname{Aut}(Q_{n+1}).\]
	Namely,
	\[\operatorname{Aut}(J_{n+1}) \cong \prod_{i=1}^{n+1} \operatorname{Aut}(Q_i).\]
	Thus, the assumption also holds for $J_{n+1}$.\\
	As a result, if $\{P_i\}_{i=1}^n$ is the collection of all the Sylow subgroups of a finite abelian group $G$, then
	\[\operatorname{Aut}(G) \cong \prod_{i=1}^{n} \operatorname{Aut}(P_i).\]
\end{proof}

\begin{align*}
\end{align*}

\end{exer}


\end{document}






